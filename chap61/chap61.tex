\chapter{情绪障碍和焦虑}

\section{情绪障碍可分为两大类:单相抑郁症和双相情感障碍}
\subsection{重度抑郁症与正常的悲伤有很大不同}
\subsection{重度抑郁症通常在生命早期开始}
\subsection{双相情感障碍的诊断需要躁狂发作}

\section{焦虑症代表恐惧回路的显着失调}
\section{遗传和环境风险因素都会导致情绪和焦虑症}
\section{抑郁和压力共享重叠的神经机制}

\section{可以通过神经影像学识别与情绪和焦虑症有关的人脑结构和回路的功能障碍}
\subsection{识别功能异常的神经回路有助于解释症状并可能提出治疗建议}
\subsection{海马体积的减少与情绪障碍有关}

\section{可以有效治疗严重的抑郁症和焦虑症}
\subsection{目前的抗抑郁药物影响单胺能神经系统}
\subsection{氯胺酮显示出作为治疗重度抑郁症的快速起效药物的前景}
\subsection{心理疗法可有效治疗重度抑郁症和焦虑症}
\subsection{电休克疗法对抑郁症非常有效}
\subsection{正在开发新形式的神经调节来治疗抑郁症}
\subsection{第二代抗精神病药物是双相情感障碍的有效治疗方法}

\section{亮点}
\subsection{选读}
\subsection{参考文献}



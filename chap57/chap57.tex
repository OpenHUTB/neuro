\chapter{周围神经和运动单位疾病}

% 参考:https://www.dxy.cn/bbs/newweb/pc/post/40268362

\section{周围神经、神经肌肉接头和肌肉的疾病可以在临床上加以鉴别}

\section{多种疾病以运动神经元和周围神经为目标}
\subsection{运动神经元疾病不影响感觉神经元(肌萎缩侧索硬化症)}
\subsection{周围神经疾病影响动作电位传导}
\subsection{一些遗传性周围神经病的分子基础已经确定}

\section{神经肌肉接头突触传递障碍有多种原因}
\subsection{重症肌无力是神经肌肉接头疾病研究最充分的例子}
\subsection{兰伯特-伊顿综合症和肉毒杆菌中毒也会改变神经肌肉传递}

\section{骨骼肌疾病可以遗传或后天获得}
\subsection{皮肌炎是获得性肌病的例证}
\subsection{肌肉萎缩症是最常见的遗传性肌病}
\subsection{一些遗传性骨骼肌疾病是由电压门控离子通道的遗传缺陷引起的}

\section{亮点}
\subsection{选读}
\subsection{参考文献}

\chapter{突触传递和神经元兴奋性的调节:第二信使}
% PDF所在目录: /data2/whd/win10/learn/neuro/neuro_神经科学原理_28_中枢神经系统的听觉处理.pdf

\section{环 AMP 通路是了解最多的由 G 蛋白偶联受体启动的第二信使信号级联}

\section{由 G 蛋白偶联受体启动的第二信使通路具有共同的分子逻辑}
\subsection{G 蛋白家族激活不同的第二信使通路}
\subsection{磷脂酶 C 水解磷脂产生两个重要的第二信使,IP3 和甘油二酯}

\section{受体酪氨酸激酶构成代谢受体的第二大家族}

\section{几类代谢物可以作为跨细胞信使}

\subsection{磷脂酶 A2 水解磷脂释放花生四烯酸以产生其他第二信使}
\subsection{内源性大麻素是抑制突触前递质释放的跨细胞信使}
\subsection{气态第二信使一氧化氮是一种刺激环 GMP 合成的跨细胞信号}

\section{代谢型受体的生理作用不同于离子型受体}
\subsection{第二信使级联可以增加或减少多种离子通道的开放}
\subsection{G蛋白可以直接调节离子通道}
\subsection{环 AMP 依赖性蛋白质磷酸化可以关闭钾通道}

\section{第二信使可以赋予突触传递持久的影响}

\section{调制器可以通过改变内在兴奋性或突触强度来影响电路功能}
\subsection{多种神经调节剂可以汇聚到同一个神经元和离子通道上}
\subsection{为什么有这么多调制器?}

\section{亮点}

\section{选读}

\section{参考文献}



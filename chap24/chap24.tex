\chapter{高层视觉处理:从视觉到认知} \label{chap:chap24}

如我们所见,低层视觉处理负责在投射到视网膜上的光模式下检测各种类型的对比度。
中层处理涉及所谓视觉元素的识别,例如轮廓和运动场,以及表面的分离。
高层视觉处理整合了来自各种来源的信息,是通向视觉知觉的最后阶段。


高层视觉处理与识别环境的有意义行为特征有关,因此取决于从短期工作记忆、长期记忆和大脑皮层执行区域传达信息的下行信号。


\section{高层视觉处理与目标识别有关}

我们对世界的\textit{视觉经验}从根本上以目标为中心。
即使投射到视网膜上的光模式随着观看条件(如光线、角度、位置和距离)的变化而变化很大,我们也能识别出同一个目标。
即使对于视觉上复杂的目标也是如此,这些目标包括大量连接的视觉特征。


此外,目标不仅仅是视觉实体,而是与特定的体验、其他记忆的目标和感觉(例如咖啡研磨机的嗡嗡声或情人香水的香气)以及各种情感相关。
目标的行为意义是基于视觉信息引导我们的行动。
简而言之,如图~\ref{fig:24_1}~所示,目标识别在视力和认知之间建立了联系。


\begin{figure}[htbp]
	\centering
	\includegraphics[width=1.0\linewidth]{chap24/fig_24_1}
	\caption{整个目标的表示是高层视觉处理的核心。
		整个目标的表示涉及视觉通路的早期阶段提取视觉特征的整合。
		这种整合是由同一目标和目标类别的不同成员生成大量视网膜图像概括。
		该表示还结合了来自其他感官模态的信息(附加情感效价),并将该目标与其他目标或事件的记忆相关联。
		目标表示可以存储在工作记忆中,并与其他记忆相关联而回忆起来。}
	\label{fig:24_1}
\end{figure}



\section{下颞皮层是目标识别的主要中心}

灵长类动物研究表明颞叶的新皮层区域(主要是下颞皮层)与目标感知有关。
由于皮层视觉系统中突触中继的层次结构从初级视觉皮层延伸到颞叶,因此颞叶是许多类型的视觉信息的汇聚点。


神经心理学研究发现,对颞叶皮层的损害会产生特定的目标识别失败。
神经生理和功能成像研究反过来又对下颞神经元的活性代表目标,这些表示与感知和认知事件的关系以及如何通过经验进行修饰的方式产生了显著的见解。


源自视网膜中的视觉信号在达到初级视觉皮层之前,在丘脑的外侧核中进行处理。
初级视觉皮层的上行视觉通路遵循 2 个主要的并行和分层组织的流:腹侧流和背侧流(第~\ref{chap:chap21}~章)。
如图~\ref{fig:24_2}~所示,腹侧流从初级视觉皮层到\textit{二级视觉皮层}向腹侧延伸到下颞皮层,在猕猴中,该皮层包括上颞沟的下部和颞叶的腹侧凸面。
该腹侧流中每个突触继转发的神经元从前阶段接收汇聚输入。
在层次结构的顶部,下颞神经元可以在广泛的视觉空间区域整合大量的视觉信息。


\begin{figure}[htbp]
	\centering
	\includegraphics[width=1.0\linewidth]{chap24/fig_24_2}
	\caption{目标识别的皮层通路。
		A. 猕猴大脑的侧视图显示了涉及视觉处理的主要通路,包括目标识别通路(红色)。
		B. 猕猴大脑的侧视图和腹侧视图显示参与目标识别的皮层区域。
		C. 下颞皮层是腹侧流的最后阶段(红色箭头) 并与内侧颞叶和前额叶皮层(灰色箭头)的相邻区域相互连接。
		此图表说明了信息流的主要连接和主要方向。}
	\label{fig:24_2}
\end{figure}


下颞皮层是大脑区域。
与该区域的解剖连接模式表明,它至少包括 2 个主要的功能细分: 后枕骨后皮层和前区域颞皮层;
功能证据表明:进一步分别属于多个功能专长的区域。
正如我们将看到的那样,神经心理学和神经生理学证据支持下颞皮层的前部和后部之间的区别。



\subsection{临床证据表明下颞皮层对于目标识别至关重要}

在19世纪后期,美国神经学家\textit{桑各$\cdot$布朗}和英国生理学家\textit{爱德华$\cdot$艾伯特$\cdot$谢弗}首次清晰地了解了介导物体识别地神经通路,他们发现:对灵长类动物颞叶的实验性损伤会丧失识别目标的能力。
与枕叶区域病变伴随的缺陷不同,颞叶病变不会损害对基本视觉属性(例如颜色、运动、距离)的敏感性。
由于视觉丧失的类型,这种损害最初被称为心理失明,但此术语后来被视觉上的\textit{失认症}取代,这是由\textit{西格蒙德$\cdot$弗洛伊德}创造的术语。


在人类中,有 2 个基本类别的视觉不适,具有敏感性和联想性,其描述导致了视觉系统中目标识别的 2 个阶段模型。
如图~\ref{fig:24_3}~所示,患有统觉失认症的人,匹配或复制复杂的视觉形状或目标的能力就会受损。
这种障碍是由于目标识别第一阶段的破坏而引起的:将视觉特征集成到整个目标的感觉表示中。
借助关联的不可思议,匹配或复制复杂目标的能力仍然完好无损,但是识别目标的能力受到损害。
这种障碍由于目标识别第二阶段的破坏而引起:目标的感官表示与目标的含义或功能的了解相关。


\begin{figure}[htbp]
	\centering
	\includegraphics[width=1.0\linewidth]{chap24/fig_24_3}
	\caption{人类颞叶中的神经元参与目标识别。
		下颞皮层受损会削弱识别视觉目标的能力,这种情况称为视觉失认症。
		有两大类视觉失认症:后部区域受损导致的知觉性失认症和前部区域受损导致的联想性失认症。}
	\label{fig:24_3}
\end{figure}


与此功能层次相一致,统觉性失认症在后下颞叶皮层受损后最常见,而联想性失认症(一种高阶知觉缺陷)在前下颞叶皮层受损后更常见。
前部分区中的神经元表现出在后部区域所没有的多种与记忆相关的特性。


颞皮层中更多的焦点病变会导致特定的缺陷。
对人类颞叶一小部分区域的损害导致无法识别面部,这是一种称为\textit{面孔失认症}的联想性失认症。
\textit{面孔失认症}患者可以将一张脸识别为脸这个类别,识别出它的各个部分,甚至可以检测到面部表达的特定情绪,但他们无法识别某张脸属于特定人。


\textit{面孔失认症}是类别特定类别的不可思议的一个例子,其中颞叶损伤的患者无法识别属于特定语义类别的特定项目。
也有报道称,患者存在针对生物、水果、蔬菜、工具或动物的特定类别的特定失认症。
由于面部的明显行为意义以及人们认识到大量面孔的正常能力,\textit{面孔失认症}可能只是最常见的特定类别特异性的\textit{失认症}。



\subsection{下颞皮层中的神经元编码复杂的视觉刺激,并按功能特化的柱进行组织}

从1970年代\textit{查尔斯$\cdot$格罗斯}和同事的工作开始,已经对颞叶中的视觉信息编码进行了广泛的研究\cite{gross1999brain}。
该区域的神经元具有独特的响应特性。
它们对简单的刺激特征(例如方向和颜色)相对不敏感。 
取而代之的是,绝大多数人具有较大的中心位置感受野,并编码复杂的刺激特征。
这些选择性通常看起来有些任意。
例如,单个神经元可能会对特定颜色和纹理的新月形模式做出强烈响应。
具有独特选择性的细胞可能会为特定有意义目标响应的高阶神经元提供输入。


实际上,如图~\ref{fig:24_4}~中\textit{查尔斯$\cdot$格罗斯}所发现的那样,在下颞皮层中,正面有意义的目标(例如面部和手)激活了几个小的神经元亚群\cite{gross1969visual}。
对于手的视觉有响应的细胞,单个手指特别关键。
在对面部有响应的细胞中,某些细胞的最有效刺激是面部的正面视图,而对于其他细胞来说,这是侧视图。
尽管某些神经元一般对面孔的响应优先,但另一些神经元仅对特定的面部表情做出响应。
这种细胞似乎直接有助于面对识别。


\begin{figure}[htbp]
	\centering
	\includegraphics[width=0.95\linewidth]{chap24/fig_24_4}
	\caption{猴子下颞皮层的神经元参与面部识别。
		A. 猴子下颞皮层的位置显示在侧视图和冠状剖面中。
		彩色区域是记录神经元的位置。 
		B. \textit{刺激时间直方图}说明了单个神经元中动作电位的频率以响应不同的图像(显示在直方图下方)。
		这个神经元选择性地对面孔做出响应。
		遮盖关键特征(例如图B中  4 和 5 的嘴巴或眼睛)会导致响应显著但并非完全减少。
		扰乱面部(2)的各个部分几乎消除了响应。}
	\label{fig:24_4}
\end{figure}


在皮层视觉系统的初始转发中,对相同刺激特征(如朝向或运动方向)做出响应但来自视野不同部分的神经元被组织成柱。
下颞皮层内的细胞具有类似的组织结构。
代表相同或相似刺激特性的神经元柱通常延伸到整个皮层厚度,并在约 400 微米的范围内延伸。
如图~\ref{fig:24_5}~所示,排列的柱是使具有某些相似特征的不同刺激在部分重叠的柱中表示。
因此,一个刺激可以激活多个柱。
水平连接可以跨越许多毫米,并可能促进用于编码目标的分布式网络的形成。


\begin{figure}[htbp]
	\centering
	\includegraphics[width=1.0\linewidth]{chap24/fig_24_5}
	\caption{下颞皮层前部对复杂视觉刺激作出响应的神经元被组织成柱。
		A. 前下颞皮层表面的光学图像说明了被右侧所示目标选择性激活的区域。
		B. 下颞皮层的神经元组织成从皮层表面延伸出来的功能特化柱。
		根据这个模型,每一柱都包含对特定视觉复杂目标做出响应的神经元。
		代表一个目标变化的神经元柱(例如不同的面孔或不同的激活熄灭器)构成了一个皮层柱。}
	\label{fig:24_5}
\end{figure}



\subsection{灵长类动物的大脑包含用于面部处理的专用系统}

面孔失认通常在没有任何其他形式失认的情况下发生。
如此高度特异性的感知缺陷可以通过位于独立簇中的面部选择性神经元的局部损伤来解释。
\textit{南希$\cdot$坎韦施}和同事使用\textit{功能性核磁共振成像}\cite{kanwisher1997fusiform}以及\textit{格雷戈里$\cdot$麦卡锡}和\textit{格雷戈里$\cdot$麦卡锡}和同事使用来自人脑表面的直接电生理记录\cite{mccarthy1997face}来加强这种想法。
\textit{坎韦施}及其同事发现,在呈现面孔和其他物体图片时,人类颞叶的一个区域(颞骨面区)在呈现面孔时响应显著高于其他物体。


随后,发现了更多的面部选择区域,主要位于颞叶,但也位于前额叶皮层中。
对这些区域的早期研究为面部选择性神经元聚集提供了间接证据。
在后来的研究中,\textit{曹颖}、\textit{温里奇$\cdot$弗赖瓦尔德}和同事直接证明了这种聚类,并表明面部处理可能由从下颞皮层后部到前额叶皮层的专用面部处理网络进行。
他们使用\textit{功能性核磁共振成像},在颞叶中发现了 6 个区域,在猕猴的前额叶皮层中发现了 3 个区域,对面部的响应比对其他目标更有选择性。
这些区域(称为\textit{面部识别块})在不同个体中位置高度一致,因此根据其位置命名。
每个面部识别块的直径为几毫米,因此在组织上与下颞皮层柱不同。
面部识别块的细胞内记录表明,绝大多数细胞对面的响应比对其他目标的响应更多。
因此,将数百万的面部细胞聚集成固定数量的小区域。
这些区域是直接连接的,从而形成了面部处理网络。
在此网络中,每个节点似乎都在功能上专业化。
从颞叶内的后部到前部,最初的\textit{面部识别块}对面部的特定视角做出响应,然后\textit{面部识别块}逐渐对身份更具选择性,而对视角的选择性更少。
此外,颞叶内的背侧面积对自然面部运动具有选择性,而腹侧面积缺乏。
因此,一个高度专业的网络主要位于颞皮层,处理面部传达的信息的多个维度(图~\ref{fig:24_6})。


\begin{figure}[htbp]
	\centering
	\includegraphics[width=1.0\linewidth]{chap24/fig_24_6}
	\caption{颞叶包含面部选择性区域网络。
		A. 猕猴观看面部和其他目标图片的功能磁共振成像识别了颞叶、\textit{颞上沟}内部和周围的 6 个面部选择性区域。
		这些区域出现在不同受试者的相同位置,并根据其解剖位置命名。
		这些区域相互连接形成一个人脸处理网络。
		B. 来自\textit{中外侧}、\textit{前外侧}和\textit{前内侧}区域的单神经元记录显示调整到头部方向。
		\textit{中外侧}细胞被调整到特定的头部方向,许多\textit{前外侧}细胞被调整到彼此镜像对称版本的多个方向,而\textit{前内侧}细胞被广泛地和更弱地调整到头部方向。
		互连区域中的这 3 种表示可以被认为是彼此的转换(箭头)。}
	\label{fig:24_6}
\end{figure}



\subsection{下颞皮层是参与目标识别皮层区域网络的一部分}

如图~\ref{fig:24_2}~所示,目标识别与视觉分类,视觉记忆和情感密切相互交织,而下颞皮层的输出会有助于这些功能。
如图~\ref{fig:24_2}C~所示,主要投射中有向临时皮层的周围和偏头皮皮层的投射,它们位于下颞皮层的腹侧表面。
这些区域反过来又投射到内嗅皮层和海马体结构,这两者都参与了长期记忆存储和检索。
下颞皮层的第二个主要投射是前额叶皮层,这是高层视觉处理的重要部位。
正如我们将看到的那样,前额叶神经元在目标分类、视觉工作记忆、记忆回忆中起着重要作用。


下颞皮层还通过外鼻周皮层直接或和间接地向杏仁核提供输入,杏仁核被认为将\textit{情绪效价}应用于感觉刺激并调动情感的认知部分和出自本能的部分(第~\ref{chap:chap42}~章)。
最后,下颞皮层是皮层多模态感觉区域的主要输入来源(例如图~\ref{fig:24_2}B~所示的上颞多感觉区,该区域位于与下颞皮层相邻的背面)。



\section{目标识别依赖于感知恒常性}

尽管有时有明显不同的视网膜图像,但在不同的观看条件下将目标识别为相同的能力是\textit{视觉经验}功能上最重要的要求之一。
目标的不变属性(例如,图像特征或特征特征(例如斑马的条纹)之间的空间和色素关系)是目标身份和含义的提示。


为了进行目标识别,这些不变属性必须独立于其他图像属性表示。
视觉系统可以熟练地做到这一点,其行为表现称为知觉恒常性。
知觉恒常性的形式从跨目标的简单转换(例如大小或位置的变化)到更困难的形式,例如深度旋转或照明的变化,甚至是类别中目标的相同性:所有斑马看起来像一样。


最好的例子之一是\textit{大小恒常性}。
与观察者不同距离处的目标被认为具有相同的大小,即使目标在视网膜上产生不同绝对大小的图像。
数百年来,大小的构成已被认可,但仅在过去几十年中,才有可能确定负责的神经机制。
一项早期的研究发现,下颞皮层的病变会导致猴子\textit{大小恒常性}的失败,这表明该区域的神经元在\textit{大小恒常性}体中起着至关重要的作用。
确实,如图~\ref{fig:24_7}A~所示,单个下颞神经元最明显的特性之一是:即使刺激大小发生很大变化,它们的形状选择性也保持不变。


知觉恒常性的另一种类型是位置稳定性,其中目标被识别为相同的位置,而不管它们在视野中的位置如何。
如图~\ref{fig:24_7}B~所示,当目标在其大的感受野内改变位置时,许多下颞神经元的选择性响应模式不会有所不同。
\textit{形式线索不变性}是指定义形式变化的线索时形式的恒定。
例如,\textit{亚伯拉罕$\cdot$林肯}头部的轮廓无论是白底黑线,黑底白线,还是绿底红线,都很容易辨认。
许多下颞神经元的响应不会随着对比极性(图~\ref{fig:24_7}C)、颜色或纹理的变化而变化。


\begin{figure}[htbp]
	\centering
	\includegraphics[width=1.0\linewidth]{chap24/fig_24_7}
	\caption{知觉恒常性反映在下颞皮层神经元的行为中。
		% ‌Lobe Stimuli‌通常指的是刺激大脑不同脑叶的音频或视觉内容。这些刺激旨在激活和测试大脑特定区域的功能。
		许多下颞神经元的响应对具有特定频率(数量)的脑叶刺激具有选择性,但不随目标大小、位置和反射率变化。
		A. \textit{大小恒常性}。
		即使视网膜图像的大小随着视野中目标的距离而减小,目标也会被感知为相同的。
		绝大多数下颞神经元对视网膜图像大小显著变化的响应是不变的,如此处单个细胞的记录所示。
		B. 位置不变。
		尽管视网膜图像中的位置发生变化,但目标仍被认为是相同的。
		几乎所有下颞神经元对视野中不同位置的相同刺激都有类似的响应,如此处单个神经元的记录所示。
		C. \textit{形式线索不变性}。
		尽管反射率发生变化,但目标仍被认为是相同的。
		如单个神经元的记录所示,大多数下颞神经元对所示的 2 个图像的响应相似。}
	\label{fig:24_7}
\end{figure}


视角不变性是指从不同角度观察到三维目标的知觉恒常性。
因为我们看到的大多数目标都是三维和不透明的,当从不同的角度看时,有些部分变得不可见,而其他部分则被揭示,而其他所有部分都会在外观变化。
然而,尽管可能由熟悉的目标施放的无限视网膜图像范围,但观察者可以轻松地独立于观察其角度识别目标。
该规则有明显的例外,通常是在从一个产生非特征性的视网膜图像的角度查看目标的情况下发生的,例如从上方直接查看的桶。


因此,目标识别机制必须从明显的复杂形状中推断目标的身份。
下颞皮层中的许多神经元都不表现出\textit{视角不变性}。
实际上,许多人系统地调整了视角。
然而,在较大的前部位置,神经元不仅大小和位置不变,而且对视角也表现出更大的不变性。
面部处理系统是一个很好的例子。
后\textit{面部识别块}中的神经元被调整为视角,而前脸斑块中的神经元表现出极大的稳健性,可以改变视点的变化。
因此,与前部区域相比,后面面部区域中的群体响应包含有关头部方向的更多信息,而前脸斑块则提供了与后面部面积相比,有关跨头方向的更多信息。
神经元的个体神经元和群体在前下颞皮层中达到的\textit{视角不变性}可能足以说明感知\textit{视角不变性}。
但这尚未直接显示。
另外,可以在较高的皮层加工阶段(例如前额叶皮层)实现\textit{视角不变性}。


研究\textit{视角不变性}失败的条件可能会导致对行为神经机制的见解。
这样的条件是镜像图像的介绍。
尽管镜像不完全相同,但经常被认为是这样的混乱,反映了系统以识别\textit{视角不变性}的错误阳性识别。
\textit{卡尔$\cdot$奥尔森}及其同事研究了下颞皮层特定区域中神经元对镜像图像的响应\cite{rollenhagen2000mirror}。
与感知混乱一致,许多下颞神经元对这 2 个图像的响应类似。
同样,在前面描述的后部和前部之间的一个面积中,剖面选择性细胞与面部的左和右谱相似。
这些结果加强了以下结论:下颞皮层中的活动反映了感知不变性,尽管在这种情况下是错误的,而不是刺激的实际特征。



\section{目标的分类感知简化了行为}

所有形式的知觉恒常物都是视觉系统试图跨单个目标产生的不同视网膜图像的尝试的产物。
更一般的恒定类型是对单个目标的感知属于同一语义类别。
例如,篮子里的苹果或不同字体的字母 A 的许多外观在物理上是不同的,但很容易被认为是完全相同的。


分类感知通常被定义为比同一类别的目标更好地区分不同类别目标的能力。
例如,在波长相差 10 纳米的 2 个红光之间进行区分比在具有相同波长差的红光和橙色光之间进行区分更困难。


分类感知简化了行为。
例如,苹果是完全球形的还是左侧略微有杂色的,还是我们提供的座椅是\textit{温莎侧椅}还是\textit{奇彭代尔式侧椅}都无关紧要。
同样,阅读能力要求一个人能够识别多种类型样式的字母。
和更简单的知觉恒常性形式一样,分类感知取决于大脑提取所见目标不变特征的能力。


是否存在一组神经元对一个类别内的目标做出一致响应,对不同类别的目标做出不同响应?
为了测试这一点,\textit{大卫$\cdot$弗雷德曼}和\textit{厄尔$\cdot$米勒}及其同事创建了一组图像,其中合并了狗和猫的特征\cite{freedman2008neural}。
复合图像中狗和猫的比例从一个极端到另一个极端变化。 
对猴子进行了训练,可以可靠地将这些刺激识别为狗或猫。
\textit{米勒}及其同事随后从背外侧前额叶皮层中的视觉响应神经元记录,该区域从下颞皮层接收直接输入\cite{miller1991neural}。
如图~\ref{fig:24_8}~所示,这些神经元不仅表现出预测的类别选择性响应(对猫的响应很好),但反之亦然,而且神经元类别边界也对应于行为学到的边界。 
相比之下,下颞皮层中的神经元代表特征的相似性,而不是类别。


\begin{figure}[htbp]
	\centering
	\includegraphics[width=1.0\linewidth]{chap24/fig_24_8}
	\caption{分类感知的神经编码。
		A. 这些图像以不同的比例组合了猫和狗的特征。
		如果图像具有猫或狗的 50\% 或更多特征,猴子会被训练将其归类为猫或狗。
		B. 周围刺激时间直方图说明\textit{前额叶皮层}神经元对 A 部分所示图像的响应。
		神经元对猫图像(100\%、80\% 和 60\%)的响应比对狗图像(60\%、80\% 和 100\%)的响应弱得多。
		尽管视网膜图像的差异与类别之间的视网膜图像差异一样大,甚至更大,但对同一类别图像的响应非常相似。
		因此,该细胞是特定于类别的。
		这种特定于类别的响应在外侧前额叶皮层的视觉神经元中很常见。}
	\label{fig:24_8}
\end{figure}


颞叶损伤后有时会出现类别特异性失认症,这一事实表明:下颞皮层的神经元与前额叶皮层的神经元具有相似的类别选择性。
颞叶皮层中的面部选择性细胞似乎符合此标准,因为它们对一系列面部的响应通常相似。
然而,这些可能构成一种特殊情况,而对于大多数刺激条件,类别选择性响应可能是前额叶皮层中神经元的特征,在额叶皮层中,视觉响应更常见于刺激的行为意义。



\section{视觉记忆是高层视觉处理的一个组成部分}

\textit{视觉经验}可以作为记忆存储,视觉记忆会影响传入的视觉信息处理。
目标识别特别依赖于观察者以前在目标上的经历。
因此,必须通过经验来修改目标识别的下颞皮层对目标识别的贡献。


关于经验在视觉感知中作用的研究集中在 2 种不同类型经验丰富的可塑性上。
一个源于反复的暴露或实践,这会改善\textit{视觉辨别}和目标识别能力。
这些依赖经验的变化构成了一种被称为感知学习的隐性学习形式(第~\ref{chap:chap23}~章)。
另一个发生在与显式学习的存储,可以有意识地回忆事实或事件的学习(第~\ref{chap:chap54}~章)。



\subsection{隐式视觉学习导致神经元响应选择性的变化}

通过经验来区分复杂视觉刺激的能力是高度可修饰的。
例如,关注汽车模型之间细微差异的个人可以提高识别这些差异的能力。


在下颞皮层中,复杂目标的神经元选择性可以发生变化,与区分目标能力的相似之处。
例如,在\textit{洛戈塞蒂斯}及其同事的一项研究中,对猴子进行了训练,可以从目标的二维视图中识别出新颖的三维目标,例如随机弯曲的线形式\cite{logothetis1995psychophysical}。
广泛的训练导致了从二维视图识别目标能力的明显提高。
如图~\ref{fig:24_9}~所示,训练后,发现一群神经元对前面的视图表现出明显的选择性,但对同一目标的其他二维视图没有明显的选择性。


\begin{figure}[htbp]
	\centering
	\includegraphics[width=1.0\linewidth]{chap24/fig_24_9}
	\caption{熟悉特定的复杂目标会导致下颞神经元对这些目标有选择性地做出响应。
		A. 训练猴子从电线的一组二维视图中识别随机弯曲的电线。 在连续的视图中,线形式旋转了 12°。
		一旦识别性能在高水平上稳定下来,就会在呈现每个视图时从下颞皮层中的神经元进行记录。
		周围刺激时间直方图显示典型神经元对每个视图的响应。
		该神经元选择性地响应代表目标小范围旋转的视图。
		B. 当用猴子不熟悉的 2 组刺激测试同一个神经元时,它对这些刺激中的任何一个都没有响应。}
	\label{fig:24_9}
\end{figure}


对猴子的其他研究表明,熟悉新面孔会改变下颞皮层中面部选择性神经元的调整。
同样,当动物具有由简单特征形成的新颖目标的经验时,下颞神经元就会对这些目标有选择性。
由于动物参与主动辨别或仅被动观察视觉刺激的原因,已经发现了这种神经元的变化,并且它们通常表现为神经选择性的锐化,而不是绝对\textit{激活率}的变化。
锐化恰恰是一种神经元变化,它可能是视觉刺激感知辨别的改善。



\subsection{视觉系统与工作记忆和长期记忆系统相互作用}

目标识别和学习是复杂的联系。
实际上,学习可以在下颞皮层中产生整个功能专业化区域。
例如,年轻学习以将特定形状(例如,数字符号)与特定奖励大小相关联的猴子会发展出处理这些特定形状的专业大脑区域。
这些大脑区域靠近前面讨论的颞叶\textit{面部识别块}。


已经研究了有关视觉与记忆之间相互作用的 2 个问题。
首先,如何在短期工作记忆中维护视觉信息?
工作记忆的容量有限,就像计算机操作系统中的缓冲区一样,整合到长期记忆中很容易受到干扰(第~\ref{chap:chap54}~章)。
第二,长期视觉记忆以及它们之间的关联是如何存储和回忆的?


在视觉延迟响应任务中,需要访问超出刺激持续时间的刺激信息(文本框~\ref{box:24_1}),在延迟期间,下颞皮层和前额叶皮层许多与视觉相关的神经元在延迟期间都在继续激活。
如图~\ref{fig:24_11}~所示,这种延迟周期活动被认为可以将信息保持在短期工作记忆中。
下颞皮层和前额叶皮层的延迟期活动在许多方面不同。
首先,下颞皮层中的活动与视觉模式和颜色信息的短期存储有关,而前额叶皮层的活动编码视觉空间信息以及从其他感觉方式中获得的信息。
下颞皮层中的延迟周期活性似乎也与视觉感知密切相关,因为它编码了样本图像,但是可以通过另一个图像的外观来消除。


\begin{proposition}[视觉与工作记忆的相互作用研究] \label{box:24_1}
	
	\quad \quad 视觉和记忆之间的关系可以通过将神经心理学方法与单细胞电生理学方法相结合来研究。
	
	\quad \quad 用于研究记忆的一种行为范式是延迟响应任务。
	受试者需要根据在短暂延迟期间记住的信息做出具体回应。
	如图~\ref{fig:24_10_a}A~所示,在这项任务的一种形式中(被称为延迟匹配样本),受试者必须指出视觉刺激与之前查看的线索刺激(样本)是相同还是不同。
	
	\quad \quad 当与单细胞记录结合使用时,这项任务允许实验者分离神经元响应的 3 个关键成分:
	(1)感觉成分,即线索刺激引起的响应;
	(2) 短期记忆或工作记忆成分,在提示和匹配之间的延迟期间发生的响应;
	以及(3)识别记忆或熟悉度分量,即由匹配刺激引起的响应与对提示刺激较早响应之间的差异。
	
	\quad \quad 第二种行为范式,视觉配对关联任务,已与电生理学结合使用,以探索关联的长期存储和回忆背后的细胞机制。
	如图~\ref{fig:24_10_b}B~所示,该任务与延迟匹配到样本任务的不同之处在于,匹配和提示是 2 种不同的刺激。
	
	\quad \quad 样本刺激可能由字母A和匹配刺激字母B组成。
	通过重复的时间配对和条件强化,受试者了解到A和B是相互预测的:它们是相关的。
	
\end{proposition}


\begin{figure}[htbp]
	\centering
	\includegraphics[width=1.0\linewidth]{chap24/fig_24_10_A}
	\caption{\textit{延迟样本匹配}任务。
		在这种范式中,试验从一个注视点的出现开始,该注视点将受试者的注意力和凝视引导到计算机屏幕的中心。
		\textit{提示}刺激(“样本”)随后短暂出现,通常持续500毫秒,随后出现一段显示为空白的\textit{延迟}。
		延迟可以变化以适应实验目标。
		延迟后,\textit{选择}显示出现,其中包含多个图像,其中一个是提示(“匹配”)。
		受试者必须通过选择提示刺激来做出\textit{响应},通常是按下按钮或对刺激进行\textit{眼跳}。
		在这里所示的任务中,所有测试图像同时出现(与样本任务同时匹配)。
		它们也可以按顺序显示(与示例任务的顺序匹配)。
		尽管对于连续任务来说,试验的持续时间可能更长,但通过限制任何时候出现的视觉刺激,这种模式对电生理学研究可能是有利的。}
	\label{fig:24_10_a}
\end{figure}


\begin{figure}[htbp]
	\centering
	\includegraphics[width=0.95\linewidth]{chap24/fig_24_10_B}
	\caption{\textit{配对偶联任务}。
		除了提示和匹配是不同的刺激之外,这种范式类似于\textit{样本匹配}范式。
		在图示的例子中,篮球是提示刺激,飞机是实验者指定的匹配刺激。
		因为这些刺激没有内在的关联,受试者必须通过试错学习来发现指定的关联。
		因此,任务是建立非同源刺激之间的联系。
		配对关联任务还可以包含样本和测试刺激之间的延迟,并且可以同时(显示)和顺序形式使用。}
	\label{fig:24_10_b}
\end{figure}


\begin{figure}[htbp]
	\centering
	\includegraphics[width=1.0\linewidth]{chap24/fig_24_11}
	\caption{当目标保存在工作记忆中时,表示目标的神经活动就会持续。 
		A. 训练猴子执行颜色匹配样本任务。
		例如,首先呈现红色刺激,然后动物必须从许多有色刺激中选择红色刺激。
		该任务在样本显示和匹配之间加入了短暂的延迟(1-2 秒),在此期间,有关正确目标颜色的信息必须保留在工作记忆中。
		猴子大脑中的紫色区域表示下颞皮层。
		B. \textit{刺激时间直方图}和动作电位光栅图说明了在延迟匹配样本任务期间下颞皮层中单个神经元的响应。
		上面的记录来自样本为红色的试验,下面的记录来自样本为绿色的试验(此处显示为蓝色)。
		记录显示细胞优先响应红色刺激。
		在使用绿色样本的试验中,神经元的活动没有改变,而在使用红色样本的试验中,细胞在样本出现后表现出短暂的活动爆发,并在整个延迟期间持续放电。
		下颞皮层和前额叶皮层中的许多视觉神经元都表现出这种行为。}
	\label{fig:24_11}
\end{figure}


相比之下,在前额叶皮层中,延迟周期活动更多地取决于任务要求,并且不会因间歇性感觉输入而终止,这表明它可能在回忆长期记忆中起作用。
\textit{厄尔$\cdot$米勒}及其同事的实验支持了这一观点\cite{miller1991neural}。
在这些实验中,对猴子进行了训练以将多对目标相关联。
然后,他们使用以下程序对他们是否学会了这些成对关联进行测试。
首先,提出了一个(示例)目标; 然后,短暂延迟后,出现了第二个(测试)目标。
指示猴子指示测试目标是否是上一次训练期间与样品配对的目标。


有 2 种可能解决此任务的方法。
在延迟期间,动物可以使用感官编码,并保留样本目标的在线表示,直到测试目标出现为止,或者可以记住样本目标的副词,并将有关关联目标的信息保留在“潜在编码”中,可能显示为测试目标的内容。
值得注意的是,在延迟期间,神经元活性似乎从一种过渡到另一个。
前额叶皮层中的神经元最初编码样本目标的感觉属性(刚刚看到的目标),但后来开始编码预期的(相关)目标。
正如我们将看到的那样,前额叶皮层中的这种前瞻性编码可能是下颞皮层的自上而下信号的来源,激活代表预期目标的神经元,从而引起对该目标的有意识回忆。


在视觉刺激之间记忆的关联的背景下,已经广泛探索了长期声明性记忆存储与视觉处理之间的关系。
一个多世纪以前,美国实验心理学学院的创始人\textit{威廉$\cdot$詹姆斯}提出:学习视觉关联可能通过编码个体刺激的神经元之间的连通性来介导。
为了检验这一假设,\textit{托马斯$\cdot$奥尔布赖特}及其同事训练猴子与没有事先物理或语义相关性的目标对相关。
后来对猴子进行了测试,同时进行了下颞皮层中神经元的细胞外记录。
已经配对的目标通常会引起类似的神经元响应,这是人们期望的,如果功能连接得到增强,而不成目标引起的响应是无关的。
如图~\ref{fig:24_12}~所示,猴子正在学习新的视觉关联时,来自单个下颞神经元的记录表明,在训练过程中,细胞对配对目标的响应变得更加相似。
最重要的是,神经元活动的变化发生在行为变化的时间表上,而神经活动的变化取决于成功学习。


\begin{figure}[htbp]
	\centering
	\includegraphics[width=0.65\linewidth]{chap24/fig_24_12}
	\caption{目标识别与联想记忆有关。
		猴子学会了视觉刺激对之间的关联,同时从下颞皮层的神经元记录活动。
		A. 为单个训练课程(572 次试验)的每个四分位数绘制配对关联任务的行为表现。
		向动物展示了 4 种新刺激(A、B、C、D),并且需要学习 2 种配对关联(A–B、C–D)。
		正如预期的那样,性能开始时是偶然的(50\% 正确),并随着动物学会这些关联而逐渐攀升。
		B. 在 A 部分描述的行为任务期间记录的下颞神经元的平均激活率。
		每条轨迹代表 4 种刺激之一(A、B、C 或 D)呈现期间的放电率。
		对所有刺激的响应在一开始都是相似的。
		随着配对关联的学习,对配对刺激 A 和 B 的神经元响应开始聚集在与对配对刺激 C 和 D 的响应不同的水平上。
		神经元的活动因此对应于两对之间的学习关联。}
	\label{fig:24_12}
\end{figure}


这些依赖于学习的颞皮层神经元刺激选择性的变化是长期的,这表明该皮层区域是关联视觉记忆神经回路的一部分。
实验结果还支持这样一种观点:即通过代表相关刺激的神经元之间突触连接强度的变化来迅速实现学习的关联。


我们知道,内侧颞叶的海马体和新皮层区域(嗅周皮层、内嗅皮层和海马旁皮层)对获得联想视觉记忆和下颞皮层的功能可塑性都至关重要。
实际上,\textit{宫下靖}及其同事的工作表明:上述成对神经元在周围皮层中比前颞皮层更为普遍\cite{miyashita1993inferior}。
因此,如图~\ref{fig:24_2}C~所示,尽管学习改变了 2 个区域中神经元的刺激选择性,但视觉相关对之间的关联从下颞皮层到周围的皮层增长。
海马体和内侧颞叶可以促进在存储关联视觉记忆所需的下颞皮层中局部神经元回路的重组。
重组本身可能是由相关视觉刺激的时间巧合引发的赫布可塑性形式(第~\ref{chap:chap49}~章)。



\section{视觉记忆的联想回忆依赖于处理视觉刺激的皮层神经元自上而下的激活}

高层视觉处理最吸引人的特征之一是:在视野中\textit{检测}图像和同一图像的\textit{回忆}在主观上相似。
前者取决于视觉信息的\textit{自下而上}流,这是我们传统上认为的视觉。
相比之下,后者是\textit{自上而下}的信息流的产物。
这种区别在解剖学上是准确的,但掩盖了以下事实:
在正常条件下,传入信号和下行信号协同产生\textit{视觉经验}。


对关联视觉记忆的研究为视觉回忆的基础机制提供了宝贵的见解。
如我们所见,视觉关联记忆通过独立代表相关刺激神经元之间的功能连通性变化来存储在视觉皮层中。
如图~\ref{fig:24_13}~所示,这一变化的实际结果是:仅在学习之前对刺激A响应的神经元在这些刺激已关联后将对A和B响应。
刺激B激活A响应性神经元可以看作是刺激A自上而下的回忆神经元相关。


\begin{figure}[htbp]
	\centering
	\includegraphics[width=1.0\linewidth]{chap24/fig_24_13}
	\caption{视觉联想和回忆的回路。
		自下而上的信号(传递观察者视野中目标信息的传入信号)被组合成下颞皮层的目标表征。
		在联想学习之前,一个神经元(蓝色)对马戏团帐篷的响应良好,但对马响应不佳。
		通过加强代表每个配对目标的神经元之间的连接(图中的间接通路),目标之间的习得关联在下颞皮层中被调节。
		因此,通过激活间接通路实现在马出现后回忆马戏团帐篷。
		间接激活也可以由工作记忆的内容(来自前额叶皮层的反馈)触发。
		在正常情况下,视觉感知是对下颞神经元的直接输入和间接输入组合的产物。}
	\label{fig:24_13}
\end{figure}


下颞皮层中的神经元正好表现出这种行为。
与提示回忆相关的活性与刺激的自下而上激活几乎相同。
这些神经生理的发现得到了许多脑成像研究的支持,这些研究在提示和自发回忆期间发现了视觉皮层中的选择性活性。


尽管图像之间学到的关联可能通过下颞皮层的回路变化来存储,但这些回路的激活以进行有意识的回忆取决于前额叶皮层的输入。
下颞皮层可能会收到一对图像之一的传入信号,并传递到前额叶皮层,其中信息将在工作记忆中维持。
如我们所见,在延迟匹配到样本任务的延迟期间,许多前额叶神经元的持续触发最初表示有关样本图像的信息,但会更改预期的相关图像。
从前额叶皮层到下颞皮层的信号将有选择地激活代表相关图像的神经元,并且该激活将构成视觉回忆的神经相关性。



\section{要点}

1.高层视觉的关键功能是目标识别。
目标识别具有含义的视觉感知。
正如著名的神经心理学家\textit{汉斯–卢卡斯$\cdot$特伯}曾经写过的那样,目标识别的失败“将以最纯粹的形式出现,作为一种正常的感知,以某种方式被剥夺了其含义。” 


2.目标识别很困难,这主要是由于外观变化、位置、距离、方向或照明条件的变化,可能会使不同目标在外观上相似。
建立模仿灵长类动物目标识别能力的计算机模型是当前和未来研究的主要挑战。


3.目标识别依赖于称为下颞皮层的颞叶区域。
达到下颞皮层的视觉信息已经通过低层视觉和中层视觉的处理机制。


4.下颞皮层的受损会导致视觉失认,这是识别目标能力的损失。
\textit{统觉性失认症}是指无法匹配或复制复杂目标,与联想性失认不同,联想性失识是指识别目标含义或功能的能力受损。
从受损或失活区域的模式中预测失认症的确切性质,从而从理解相关性到神经目标表征的原因,是目标识别和神经学领域的主要目标。 


5.下颞皮层中的单个细胞有高度形状选择性,并有选择地对手或面部响应。
他们可以在位置、大小甚至旋转之间保持选择性(这些特性可能解释知觉恒常性)。


6.下颞皮层包括迄今尚未知数量的功能特化非常不同的区域。
尽管整个组织的功能逻辑尚不清楚,但我们确实知道具有相似选择性组的细胞凝聚成皮层柱,并且面部细胞被组织成称为面部区域的较大单元。


7.面部识别受到多个面部区域的支持,每个区域都有独特的功能专业化。
面部区域有选择性地结合在一起,形成面部处理网络,该网络已成为高层视觉的模型系统。


8.下颞皮层与鼻周皮层和海马旁回皮层相互连接,用于记忆形成,与杏仁核相连,赋予目标情感价值,以及用于目标分类和视觉工作记忆的前额叶皮层。
如果将关联记忆存储为神经元之间的连接模式,那么海马体和内侧颞叶的新皮层结构的具体贡献是什么,以及通过哪种细胞机制施加影响?
分子遗传学,细胞,神经生理学和行为学方法的汇合有望解决这些问题和其他问题。


9.目标被视为类别的成员。
这简化了适当行为的选择,这通常不取决于刺激细节。
具有分类选择性的神经元在背侧前额叶皮层(下颞皮层的主要投射位点)中发现。


10.目标识别取决于过去的经验。
感知学习可以提高区分复杂目标和下颞皮层中的神经选择性的能力。


11.可以在短期工作记忆中保存视觉信息,以超过感觉刺激的持续时间。
刺激消失后,颞叶皮层和前额叶皮层的神经元可以表现出延迟激活。
这些网络如何建立保持信息在线的能力是一个悬而未决的问题。


12.高层视觉信息处理随着自上而下的调制变化。
视觉图像的感觉体验和记忆中相同刺激的回忆在主观上相似。
下颞皮层中的神经元在自下而上激活和提示回忆中表现出相似的激活。



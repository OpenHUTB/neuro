\chapter{修复受损的大脑}


\section{轴突的损伤会影响神经元和邻近细胞}
\subsection{轴突变性是一个活跃的过程}
\subsection{轴突切开导致附近细胞的反应性反应}

\section{受伤后中央轴突再生不良}

\section{治疗干预可能促进受伤中枢神经元的再生}
\subsection{环境因素支持受伤轴突的再生}
\subsection{髓磷脂的成分抑制神经突生长}
\subsection{损伤引起的疤痕阻碍轴突再生}
\subsection{内在增长计划促进再生}
\subsection{完整轴突形成新的连接可导致损伤后功能的恢复}

\section{受伤大脑中的神经元死亡,但可以产生新的神经元}

\section{治疗干预可能会保留或替换受伤的中枢神经元}
\subsection{神经元或其祖细胞的移植可以替代丢失的神经元}
\subsection{刺激损伤区域的神经发生可能有助于恢复功能}
\subsection{非神经元细胞或其祖细胞的移植可以改善神经元功能}
\subsection{功能恢复是再生疗法的目标}

\section{要点}
\subsection{选读}
\subsection{参考文献}


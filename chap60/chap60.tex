\chapter{精神分裂症的思想和意志障碍}

在本章和下一章中,我们将研究影响知觉、思想、情绪、情绪和动机的疾病:精神分裂症、抑郁症、双相情感障碍和焦虑症。 这些一直很难理解,但最近在遗传分析方面的进展已经开始为它们的发病机制提供重要线索。

精神疾病对个人、家庭和社会都有破坏性影响。 世界卫生组织报告说,总体而言,精神疾病构成了全世界残疾的主要原因,并且是世界卫生组织报告的每年 800,000 例自杀的主要风险因素。 此外,抑郁症和焦虑症经常与糖尿病、冠状动脉疾病、中风和其他几种疾病同时发生并恶化其结果。

20 世纪中叶发现的抗精神病药、锂和抗抑郁药等药物使关闭大型且通常不合格的精神病院成为可能; 然而,中途宿舍和其他限制较少的治疗场所并没有足够多地实现。 结果,许多患有精神分裂症和严重双相情感障碍的人在一生中的某个时候变得无家可归,而且在许多国家,患有严重精神障碍的人占监狱人口的很大一部分。

此外,虽然抗精神病药物、锂盐和抗抑郁药物在控制精神障碍症状方面发挥了重要作用,但治疗效果仍然存在很大局限性。 例如,对于高度致残的认知障碍和精神分裂症的缺陷症状没有有效的治疗方法。 即使对于可以从现有药物中获益的症状,如幻觉和妄想,残留症状仍然存在并且复发是常态。 由于人脑带来的重大科学挑战和精神障碍动物模型的局限性,50 多年来精神药物的疗效几乎没有进步。 然而,人类遗传学和神经科学的最新进展为改善这种不幸的状况创造了重要机会。

\section{精神分裂症的特征是认知障碍、缺陷症状和精神病症状}
在医学中,对疾病的理解及其诊断最终基于两个特征的识别:(1) 病因学因素(例如,微生物、毒素或遗传风险)和 (2) 发病机制(通过 哪些病原体会导致疾病)。 虽然人类遗传学和神经科学开始深入了解精神分裂症、双相情感障碍和自闭症谱系障碍等疾病的病因和发病机制,但这项研究尚未产生客观的诊断测试或生物标志物。 因此,精神病学诊断仍然依赖于对患者症状的描述、检查者的观察以及随着时间推移的病程。

精神分裂症是一种非常严重的疾病。 其症状可分为三类:(1)认知症状; (2) 缺陷或阴性症状; (3) 精神症状。 这些症状群表现出不同的发作时间模式——认知障碍和缺陷症状通常最早出现。 不同的发病时间和每个簇的确切症状被认为是由发育致病机制对不同神经回路和大脑区域的影响造成的。 因此,作用于疾病过程的一个“下游”方面的抗精神病药物等现有治疗方法对认知障碍或缺陷症状没有任何有益影响。

20 世纪初,德国的 Emil Kraepelin 认识到认知能力下降是精神分裂症的一个显着特征,因为精神病症状会出现在多种精神疾病中。 事实上,Kraepelin 对后来被称为精神分裂症的术语是早发性痴呆,这个术语强调了认知丧失的早期发作。 精神分裂症的认知障碍针对工作记忆和执行功能、陈述性记忆、语言流畅性、识别面部表情所传达的情绪的能力以及社会认知的其他方面。 现有药物并不能显着改善这些损伤,但正在进行的研究表明,以认知矫正为目标的心理疗法带来的益处虽然不大,但很有前景。

缺陷症状包括情绪反应迟钝、社交互动退缩、思想和言语内容贫乏以及动力丧失。 精神病症状包括幻觉、妄想和思维紊乱,例如联想松弛(方框 60-1)。 精神分裂症的精神病症状对抗精神病药物有反应。 这些药物还可以减少其他神经精神疾病中出现的精神病症状,包括双相情感障碍、严重抑郁症和神经退行性疾病,如帕金森病、亨廷顿病和阿尔茨海默病。

\subsection{精神分裂症具有在生命的第二个和第三个十年发病的特征性病程}
精神分裂症影响全球 0.25\% 至 0.75\% 的人口,区域差异很小。 男性比女性更容易受到影响,性别比例估计为 3:2,而且男性发病通常较早。 精神分裂症通常开始于青少年后期或二十出头至中期。 持续的认知和缺陷症状通常在精神病症状出现前几个月甚至几年就开始了。 这一时期被一些研究人员称为超高风险状态,而其他人则称为精神分裂症前驱症状。

处于这种风险状态的个人通常会出现可测量的认知功能下降,并伴有社会孤立、多疑以及参与学校工作或其他任务的动力下降等症状。 减轻的精神病性症状通常随之而来,包括短暂的和轻度的幻觉。 并非每个有此类症状的青少年都会发展出所有症状,从而诊断出精神分裂症。 一小部分恢复; 其他人则发展出精神分裂症以外的严重精神疾病。 抗精神病药物似乎不会使处于危险状态的个体受益,也不会延缓精神分裂症的发作。 然而,谈话疗法和通过基于计算机的方法提供的旨在认知矫正的疗法显示出延迟精神病发作的希望。

\subsection{精神分裂症的精神病症状往往是发作性的}
精神症状,包括幻觉和妄想,是精神分裂症最显着的表现。 幻觉是在没有适当的感觉刺激的情况下发生的知觉,它们可能发生在任何感觉方式中。 在精神分裂症中,最常见的幻觉是听觉幻觉。 通常,受影响的人会听到声音,但噪音和音乐也很常见。 有时,这些声音会进行对话,并且经常被视为贬损或欺凌。 偶尔,声音会向受影响的个人发出命令,这可能会对自己或他人造成伤害的高风险。

妄想是没有现实基础的坚定信念,不能用患者的文化来解释,也不能通过论证或证据来改变。 妄想的形式可能千差万别。 对于一些受影响的人来说,现实被严重扭曲:世界充满了只为受影响的人准备的隐藏迹象(参考想法),或者这个人认为他正在被密切监视、跟踪或迫害(偏执妄想)。 其他人可能会经历奇怪的错觉; 例如,他们可能认为有人正在将思想插入他们的思想或从他们的思想中提取思想,或者他们的近亲已被来自另一个星球的外星人所取代。 除了患者持久的认知障碍外,精神病发作还经常伴有思维紊乱和言语古怪(方框 60-1)。

精神病症状也可能发生在其他神经精神疾病中,例如双相情感障碍、重度(单相)抑郁症、各种神经退行性疾病和药物诱发的状态。 然而,通常可以通过相关症状和发病年龄将这些其他病症与精神分裂症区分开来。 一旦精神分裂症变得完全明显,精神病症状往往是偶发的。 伴有显着紊乱的思维、情绪和行为的严重精神病期穿插于精神病性症状较轻甚至不存在的时期。 精神病发作通常需要住院治疗; 抗精神病药物可显着缩短此类发作的严重程度和持续时间。 第一次和第二次精神病发作通常对抗精神病药物有完全反应,但认知障碍和缺陷症状通常会持续存在。 在最初几次精神病复发后,精神分裂症患者通常会在急性复发之间出现残留的精神病症状,并且尽管接受了抗精神病药物治疗,但仍会出现这些症状。 认知和社会功能通常会在数年内持续恶化,直到达到远低于患者病前功能水平的稳定水平。


\section{精神分裂症的风险受基因的高度影响}
早在 1930 年,德国的 Franz Kalman 就研究了精神分裂症的家族模式,并得出结论认为基因起着重要作用。 为了更清楚地将遗传影响与环境影响区分开来,Seymour Kety、David Rosenthal 和 Paul Wender 在丹麦对出生时或出生后不久被收养的儿童进行了检查。 他们发现,与收养家庭的精神分裂症发病率相比,被收养人的生物学家庭中的精神分裂症发病率更能预测精神分裂症。

Kety 和他的同事还观察到,一些患有精神分裂症的被收养人的生物学亲属表现出与精神分裂症相关的较轻的症状,例如社会孤立、多疑、古怪的信念和神奇的思维,但不是坦率的幻觉或妄想。 自 Kety 时代以来,人们观察到这些亲属也可能表现出介于未受影响个体和精神分裂症患者之间的认知障碍。 他们还可能表现出通过磁共振成像 (MRI) 观察到的大脑皮层变薄,这也介于健康个体和精神分裂症患者之间。 (精神分裂症的皮层变薄将在下文讨论。)这些人现在被诊断出患有精神分裂症,这似乎是精神分裂症谱系中较温和的一端。 症状的严重程度和性质似乎受到个人风险相关遗传变异的总体负担以及暴露于环境风险因素的影响。

欧文·戈特斯曼 (Irving Gottesman) 对丹麦精神分裂症患者扩展系谱的研究支持了基因的重要性。 Gottesman 注意到亲属患精神分裂症的风险与他们与受影响的人共享 DNA 序列的程度之间的相关性。 他发现一级亲属(包括父母、兄弟姐妹和孩子,他们与患者有 50\% 的 DNA 序列)比二级亲属(包括阿姨、叔叔、侄女、侄子和孙辈)患精神分裂症的终生风险更高, 他们共享 25\% 的 DNA 序列)。 即使是三级亲属(仅共享患者 12.5\% 的 DNA 序列)患精神分裂症的风险也高于一般人群中约 1\% 的患此病风险(图 60-1)。

根据 Gottesman 在这些谱系中测得的风险水平差异,他认识到精神分裂症风险不会像孟德尔显性或隐性特征那样在家族内传播(即,它不是由单一基因位点引起的)。 他正确地预测,精神分裂症是一种多基因特征,涉及整个人类基因组中的大量位点。 这种遗传结构是许多人类表型(包括疾病表型)的基础,并且可能涉及基因组中的数百个位点。 在多基因性状中,每个与疾病相关的位点的变异对表型的贡献很小,且具有累加效应。 遗传风险变异与环境因素共同作用产生精神分裂症表型。

2014 年,一个大型全球财团报告了一项针对超过 35,000 名精神分裂症患者的全基因组关联研究。 该研究确定了 108 个与精神分裂症相关的全基因组重要位点,这些位点分布在整个基因组中。 研究仍在继续,已知基因座的数量已经超过 250 个。这些基因座中的每一个都代表一个由单核苷酸多态性识别的 DNA 片段,它赋予精神分裂症风险的小幅增加(通常为 5\%–10\%)。 这种等位基因变异的价值在于作为一种工具来识别在疾病分子机制中起作用的基因。 反过来,相关基因有助于识别可能在治疗药物开发中被利用的分子途径。


\section{精神分裂症的特点是大脑结构和功能异常}
\subsection{大脑皮层中灰质的丢失似乎是由突触联系的丢失而不是细胞的丢失引起的}
\subsection{青春期大脑发育异常可能导致精神分裂症}

\section{抗精神病药物作用于大脑中的多巴胺能系统}

\section{亮点}
\subsection{选读}
\subsection{参考文献}

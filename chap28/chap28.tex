\chapter{听觉处理的中枢神经系统}
% PDF所在目录: /data2/whd/win10/learn/neuro/neuro_神经科学原理_28_中枢神经系统的听觉处理.pdf

\section{声音向有听觉的动物传达多种类型的信息}

\section{中央通路中声音的神经表征始于耳蜗核}
\subsection{耳蜗神经以平行途径将声学信息传递到音调组织的耳蜗核}
\subsection{耳蜗腹核提取有关声音的时间和频谱信息}
\subsection{耳蜗背核将声学与体感信息相结合,利用频谱线索定位声音}

\section{哺乳动物的上橄榄复合体包含用于检测耳间时间和强度差异的独立电路}
\subsection{内侧上橄榄生成耳间时差图}
\subsection{外侧上橄榄检测耳间强度差异}
\subsection{上橄榄复合体向耳蜗提供反馈}
\subsection{下丘外侧丘系形状响应的腹侧核和背侧核抑制}

\section{传入听觉通路在下丘汇聚}
\subsection{来自下丘的声音位置信息在上丘中创建声音的空间图}

\section{下丘传输声音信息给大脑皮层}
\subsection{沿着上行通路刺激选择性逐渐增加}
\subsection{听觉皮层映射众多的声音方位}
\subsection{从下丘而来的第二声音定位通路涉及凝视控制的大脑皮层}
\subsection{大脑皮层中的听觉回路被分离成分开的处理流}
\subsection{大脑皮层在皮下听区调制感觉加工}

\section{大脑皮层形成复杂的声音表示}
\subsection{听觉皮层使用时间和速率编码来表征时变声音}
\subsection{灵长类有专门的皮层神经元编码音高和泛音}
\subsection{食虫蝙蝠有皮层区域专门负责行为相关的声音特征}
\subsection{听觉皮层涉及处理说话时的声音反馈}

\section{要点}
\subsection{选读}
\subsection{参考文献}

\chapter{自发运动:运动皮层}
% PDF所在目录: /data2/whd/win10/learn/neuro/neuro_神经科学原理_28_中枢神经系统的听觉处理.pdf

\section{自愿运动是行动意图的身体表现}
\subsection{理论框架有助于解释行为和自愿控制的神经基础}
\subsection{许多额叶和顶叶皮层区域参与自愿控制}
\subsection{下行运动命令主要由皮质脊髓束传递}
\subsection{在运动开始之前施加一个延迟期将与计划相关的神经活动与执行动作相关的神经活动隔离开来}

\section{顶叶皮层提供有关世界和身体的信息,用于状态估计以计划和执行电机动作}
\subsection{顶叶皮层将感觉信息与运动动作联系起来}
\subsection{身体位置和运动在后顶叶皮层的几个区域表示}
\subsection{空间目标在后顶叶皮层的几个区域都有体现}
\subsection{内部产生的反馈可能影响顶叶皮层活动}

\section{前运动皮层支持运动选择和规划}
\subsection{内侧前运动皮层参与自主行为的情境控制}
\subsection{背侧前运动皮层参与规划手臂的感觉引导运动}
\subsection{背侧前运动皮层参与应用管理行为的规则(关联)}
\subsection{腹侧前运动皮层参与规划手的运动动作}
\subsection{前运动皮层可能有助于指导运动行为的知觉决策}
\subsection{当观察到其他人的运动动作时,几个皮层运动区会活跃}
\subsection{自主控制的许多方面分布在顶叶和前运动皮层}

\section{初级运动皮层在运动执行中起着重要作用}
\subsection{初级运动皮层包括运动外围的详细地图}
\subsection{初级运动皮层中的一些神经元直接投射到脊髓运动神经元}
\subsection{初级运动皮层的活动反映了运动输出的许多时空特征}
\subsection{初级运动皮层活动也反映了运动的高阶特征}
\subsection{感觉反馈迅速传递到初级运动皮层和其他皮层区域}
\subsection{初级运动皮层是动态的和适应性强的}

\section{要点}
\subsection{荐读}
\subsection{参考文献}
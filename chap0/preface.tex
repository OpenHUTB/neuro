
% 参考: https://zhuanlan.zhihu.com/p/273186198
%\label{chap:preface}
%\begin{table}[htbp]
%	\newcommand{\tabincell}[2]{\begin{tabular}{@{}#1@{}}#2\end{tabular}} %换行指令
%	\centering
%	\caption{名词列表 \label{tab:0_1}}
\renewcommand\arraystretch{1.0}	%设置表格内行间距
%\setlength{\tabcolsep}{4.5mm}{
\begin{longtable}{lll}
% https://blog.csdn.net/maths_girl/article/details/107030167
\caption{名词中英对照表 \label{tab:0_1}} \\
	\toprule 
 英文(缩略词)   && 中文 \\
 
 	\midrule
 	5-hydroxyindoleacetic acid (5-HIAA)     && \href{https://baike.baidu.com/item/5-%E7%BE%9F%E5%9F%BA%E5%90%B2%E5%93%9A%E4%B9%99%E9%85%B8/16984024}{5-羟基吲哚乙酸}    \\
 	
 	\midrule
 	5-hydroxytryptophan (5-HT)     && \href{https://baike.baidu.com/item/5-%E7%BE%9F%E5%9F%BA%E8%89%B2%E6%B0%A8%E9%85%B8/5687636}{5-羟基色氨酸}    \\
 
 	\midrule
 	abducens     && 外旋神经   \\
 
 	\midrule
 	abduction     && 外转(角膜向外的眼球运动)   \\
 
	\midrule
	acetylcholine (ACh)     && 乙酰胆碱   \\
	
	\midrule
	adduction     && 内转   \\
	
	\midrule
	adrenocorticotropic hormone (ACTH)     && \href{https://baike.baidu.com/item/%E4%BF%83%E8%82%BE%E4%B8%8A%E8%85%BA%E7%9A%AE%E8%B4%A8%E6%BF%80%E7%B4%A0/2388734}{促肾上腺皮质激素}   \\
	
	\midrule
	agonist–antagonist     &&  兴奋-拮抗  \\
	
	\midrule
	agrin     &&  聚集蛋白  \\
	
	\midrule
	amacrine cell     && 无长突细胞   \\
	
	\midrule
	amygdala; amygdaloid     && 杏仁核   \\
	
	\midrule
	Angelman Syndrome     && \href{https://baike.baidu.com/item/%E5%A4%A9%E4%BD%BF%E7%BB%BC%E5%90%88%E5%BE%81/4662845}{天使综合症}   \\
	
	\midrule
	annulus of Zinn     && 总腱环   \\
	
	\midrule
	anterior cingulate cortex (ACC)     && 前扣带皮层   \\
	
	\midrule
	anterior commissure (AC)     && 前连合   \\
	
	\midrule
	anterior insula (AI)     && 前脑岛   \\
	
	\midrule
	anterior thalamus (antTHAL)     && 前丘脑   \\
	
	\midrule
	Asperger syndrome     && 阿斯伯格综合症   \\
	
	\midrule
	autism spectrum disorder (ASD)     && \href{https://baike.baidu.com/item/%E8%87%AA%E9%97%AD%E7%97%87%E8%B0%B1%E7%B3%BB%E9%9A%9C%E7%A2%8D/1704369}{自闭症谱系障碍}   \\
	
	\midrule
	autistic savant     && 自闭症学者   \\
	
	\midrule
	Basal temporal   && 基底颞叶  \\
 
	\midrule
	Best Frequency (BF)     && 最佳频率   \\
 
	\midrule
	Boold Oxygen-Level Dependent (BOLD)     && 血氧水平依赖   \\
	
	\midrule
	BRAIN Initiative     && 脑计划   \\
	
	\midrule
	brain stem (BS)     && 脑干   \\
	
	\midrule
	Brodmann area (Cg25)   && 布罗德曼 25 区  \\
	
	\midrule
	choline acetyltransferase (ChAT)   && 胆碱乙酰转移酶  \\
	
	\midrule
	Cochlear Nucleus(CN)   && 耳蜗核  \\
	
	\midrule
	Constant-Frequency (CF)     &&  恒频  \\
	
	\midrule
	cone cell      && 视锥细胞  \\
	
	\midrule
	cortical     &&  皮层  \\
	
	\midrule
	corticotropin-releasing hormone (CRH)    &&  \href{https://baike.baidu.com/item/%E4%BF%83%E8%82%BE%E4%B8%8A%E8%85%BA%E7%9A%AE%E8%B4%A8%E6%BF%80%E7%B4%A0%E9%87%8A%E6%94%BE%E6%BF%80%E7%B4%A0/3760624}{促肾上腺皮质激素释放激素}  \\
	
	\midrule
	DeoxyriboNucleic Acid (DNA)     &&  脱氧核糖核酸  \\
	
	\midrule
	Deep brain stimulation (DBS)     &&  脑深部电刺激  \\
	
	\midrule
	depression, infraduction     &&  下转(角膜向下的眼球运动)  \\
	
	\midrule
	Diagnostic and Statistical Manual of 	Mental Disorders     &&  《精神疾病诊断和统计手册》  \\
	
	\midrule
	Diagnostic and Statistical Manual of 	Mental Disorders, Fifth Edition (DSM-5)     &&  《精神疾病诊断和统计手册(第五版)》  \\
	
	\midrule
	Doppler-shifted constant-frequency (DSCF)     &&  多普勒频移恒频  \\
	
	\midrule
	dorsal anterior cingulate cortex (dACC)     &&  背前扣带皮层  \\
	
	\midrule
	dorsal caudate nucleus (dCN)     &&  背侧尾状核  \\
	
	\midrule
	dorsolateral prefrontal cortex (DLPFC, F9)     &&  背外侧前额叶皮层  \\
	
	\midrule
	dorsomedial prefrontal cortex (DMPFC)     &&  背内侧前额叶皮层  \\
	
	\midrule
	dorsomedial thalamus (dmTHAL)     &&  背内侧丘脑  \\
	
	\midrule
	electroconvulsive therapy (ECT)     &&  电痉挛疗法  \\
	
	\midrule
	electrocorticographical (ECoG)     &&  脑电图  \\
	
	\midrule
	elevation, supraduction     &&  上转(角膜向上的眼球运动)  \\
	
	\midrule
	Excitatory PostSynaptic Potential (EPSP)     &&  兴奋性突触后电位  \\
	
	\midrule
	Excited-Excited neuron (EE neuron)     &&  双耳兴奋神经元  \\
	
	\midrule
	Excited-Inhibited neuron (EI neuron)     && 兴奋-抑制神经元   \\
	
	\midrule
	extorsion     && 外旋   \\
	
	\midrule
	Extrastriate     && 纹外皮层   \\
	
	\midrule
	formant frequencies     &&  共振峰频率  \\
	
	\midrule
	fragile X syndrome     &&  \href{https://baike.baidu.com/item/%E8%84%86%E6%80%A7X%E7%BB%BC%E5%90%88%E5%BE%81/12612308}{脆性X综合症}  \\
	
	\midrule
	frequency-modulated (FM)     &&  调频  \\
	
	\midrule
	frontoinsular cortex (FI)     &&  前岛叶皮层  \\
	
	\midrule
	functional magnetic resonance imaging (fMRI)     &&  功能性磁共振成像  \\
	
	\midrule
	fusiform gyrus (FG)     &&  梭状回  \\
	
	\midrule
	Generalized anxiety disorder (GAD)     &&  广泛性焦虑障碍  \\
	
	\midrule
	genome-wide association studies (GWAS)     &&  全基因组关联研究  \\
	
	\midrule
	George Widener     &&  \href{https://baike.baidu.com/item/%E4%B9%94%E6%B2%BB%C2%B7%E6%80%80%E5%BE%B7%E7%BA%B3/58006951}{乔治·怀德纳}  \\
	
	\midrule
	highfunctioning autism     &&  高功能自闭症  \\
	
	\midrule
	hypothalamic–pituitary–adrenal (HPA)     &&  下丘脑-垂体-肾上腺  \\
	
	\midrule
	hypothalamus (HT)     &&  下丘脑  \\
	
	\midrule
	inferior frontal gyrus (IFG)   && 额下回  \\
	
	\midrule
	inferior oblique   && 下斜肌  \\
	
	\midrule
	inferior posterior regions of prefrontal cortex (IPPFC)  && 后下部前额叶皮层  \\
	
	\midrule
	inferior rectus   && 下直肌  \\
	
	\midrule
	inferior rectus   && 下直肌  \\
	
	\midrule
	insula (Ins)   && 脑岛  \\
	
	\midrule
	Intelligence Quotient (IQ)   && 智商  \\
	
	\midrule
	interstitial nucleus of Cajal (iC)   && 间位核  \\
	
	\midrule
	interstitial nucleus of the medial longitudinal fasciculus   && 内侧纵束间质核  \\
	
	\midrule
	intorsion   && 内旋  \\
	
	\midrule
	lateral parietal   && 顶叶外侧  \\
	
	\midrule
	lateral rectus   && 外直肌  \\
	
	\midrule
	Lateral Superior Olivary(LSO)   && 外侧上橄榄  \\
	
	\midrule
	lateral view   && 侧视图  \\
	
	\midrule
	levator   && 眼提肌  \\
	
	\midrule
	low-density lipoprotein receptor-related protein 4 (LRP4)   && 低密度脂蛋白受体相关蛋白  \\
		
	\midrule
	Magnetoencephalography (MEG)   && 脑磁图  \\
	
	\midrule
	major depression   && 重度抑郁症  \\
	
	\midrule
	major histocompatibility (MHC)   && 主要组织相容性  \\
	
	\midrule
	Medial Geniculate Body (MGB)   && 内侧膝状体  \\
	
	\midrule
	medial frontal cortex (mF10)   && 内侧前额叶皮层  \\
	
	\midrule
	medial longitudinal fasciculus   && 内侧纵束  \\
	
	\midrule
	Medial Nucleus of the Trapezoid Body(MNTB)   && 斜方体内侧核  \\
	
	\midrule
	Medial prefrontal cortex    && 内侧前额叶皮层  \\
	
	\midrule
	medial rectus    && 内直肌  \\
	
	\midrule
	Medial Superior Olive(MSO)   && 内侧上橄榄  \\
	
	\midrule
	mesencephalic reticular formation   && 中脑网状结构  \\
	
	\midrule
	middle cingulate cortex (MCC)   && 中扣带皮层  \\
	
	\midrule
	mid-subcallosal cingulate (Mid-SCC)  && 中下胼胝体扣带皮层  \\
	
	\midrule
	monoamine oxidase (MAO)   && 单胺氧化酶  \\
	
	\midrule
	muscle-specific trk-related receptor with a
	kringle domain(MuSK)   && 跨膜受体蛋白酪氨酸激酶  \\
	
	\midrule
	Müllerian inhibiting substance (MIS)   && 缪勒管抑制物  \\
	
	\midrule
	Müller's muscle   && 米勒肌  \\
	
	\midrule
	mygdala     && 杏仁核   \\
	
	\midrule
	Neuroligins (NLs)   && 神经连接蛋白 \\
	
	\midrule
	noradrenergic (NA)   && 去甲肾上腺素能 \\
	
	\midrule
	norepinephrine transporter (NET)   && 去甲肾上腺素转运蛋白 \\
	
	\midrule
	orbital frontal cortex (OFC, OF11)   && 眶额皮层 \\
	
	\midrule
	nucleus of Darkshevich   && 达克谢维奇核  \\
	
	\midrule
	N-Methyl-D-Aspartate (NMDA)   && N-甲基-D-天冬氨酸  \\
	
	\midrule
	oculomotor     && 动眼神经   \\
	
	\midrule
	optic nerve     && 视神经   \\
	
	\midrule
	outsider artist     && 世外艺术家   \\
	
	\midrule
	paramedian pontine reticular formation     && 桥旁正中网状结构   \\
	
	\midrule
	periaqueductal gray matter     && 中脑导水管周围灰质   \\
	
	\midrule
	photoreceptors     && 光感受器   \\
	
	\midrule
	polygenic risk scores (PRS)     && 多基因风险评分   \\
	
	\midrule
	positron emission tomography (PET)     && 正电子发射断层成像   \\
	
	\midrule
	posterior commissure     && 后连合   \\
	
	\midrule
	Posterior Parietal Cortex (PPC)     && 后顶叶皮层   \\
	
	\midrule
	pontine nuclei (PN)    && 	脑桥核   \\
	
	\midrule
	posttraumatic stress disorder (PTSD)     && 	创伤后应激障碍   \\
	
	\midrule
	Prader-Willi syndromes     && 	\href{https://baike.baidu.com/item/%E5%B0%8F%E8%83%96%E5%A8%81%E5%88%A9%E7%97%87/7472495}{小胖威利症}   \\
	
	\midrule
	prefrontal cortex (F46)     && 	前额叶皮层   \\
	
	\midrule
	pre-supplementary motor area (Pre-SMA)     && 	前辅助运动区   \\
	
	\midrule
	Rapsyn   && 突触受体相关蛋白  \\
	
	\midrule
	primary auditory cortex (A1)   && 初级听觉皮层  \\
	
	\midrule
	putamen (Put)   && 壳核  \\
	
	\midrule
	Receptive Field (RF)   && 感受野  \\
	
	\midrule
	rectus muscle   && 直肌  \\
	
	\midrule
	Rett syndrome   && \href{https://baike.baidu.com/item/%E9%9B%B7%E7%89%B9%E9%9A%9C%E7%A2%8D/22296155}{雷特综合症}  \\
	
	\midrule
	rod bipolar   && 杆状双极细胞  \\
	
	\midrule
	rod cell   && 视杆细胞  \\
	
	\midrule
	Rostral auditory cortex (R)   && 嘴侧听觉皮层  \\
	
	\midrule
	Rostrotemporal auditory cortex (R)   && 前颞听觉皮层 \\
	
	\midrule
	Sally-Anne test   && 萨莉-安妮测试 \\
	
	\midrule
	savant syndrome   && \href{https://baike.baidu.com/item/%E5%AD%A6%E8%80%85%E7%BB%BC%E5%90%88%E7%97%87/4453123}{学者综合症} \\
	
	\midrule
	selective serotonin reuptake inhibitors   && 选择性血清再吸收抑制剂 \\
	
	\midrule
	Schwann cell   && 施旺细胞 \\
	
	\midrule
	serotonin reuptake transporter   && 血清素再摄取转运蛋白 \\
	
	\midrule
	sex-determining region on Y (SRY)   && Y染色体性别决定区 \\
	
	\midrule
	shell shock   && 战斗疲劳症 \\
	
	\midrule
	Social anxiety disorder   && 社交焦虑症 \\
	
	\midrule
	Striatum   && 纹状体 \\
	
	\midrule
	STS-temporalparietal junction   && 颞上沟-颞顶联合区 \\
	
	\midrule
	sublenticular extended amygdala (SLEA)   && 近管状延伸杏仁核 \\
	
	\midrule
	substantia nigra and ventral tegmental area of the midbrain (SN/VTA)   && 黑质/中脑腹侧被盖区 \\
	
	\midrule
	superior oblique   && 上斜肌 \\
	
	\midrule
	superior rectus   && 上直肌 \\
	
	\midrule
	superior temporal sulcus (STS)   && 颞上沟 \\
	
	\midrule
	superior view   && 俯视图 \\
	
	\midrule
	supplementary motor area (SMA)   && 辅助运动区 \\
	
	\midrule
	temporal pole   && 颞极  \\
	
	\midrule
	theory of mind   && \href{https://baike.baidu.com/item/%E5%BF%83%E6%99%BA%E7%90%86%E8%AE%BA/8719175}{心智理论}   \\
	
	\midrule
	tonotopic map   && 音调拓扑图  \\
	
	\midrule
	Transcranial magnetic stimulation (TMS)   && 经颅磁刺激  \\
	
	\midrule
	transverse temporal gyri (Heschl's gyrus)   && 颞横回  \\
	
	\midrule
	trochlear   && 滑车神经  \\
	
	\midrule
	V1   && 初级视觉皮层  \\
	
	\midrule
	ventral caudate (vCD)   && 腹侧尾状核  \\
	
	\midrule
	Ventral Nucleus of the Trapezoid Body(MNTB)   && 斜方体腹侧核  \\
	
	\midrule
	ventrolateral prefrontal cortex (VLPFC, F47)   && 腹外侧前额叶皮层  \\
	
	\midrule
	vergence movement   && 聚散运动  \\
	
	\midrule
	version movement   && 同向运动  \\
	
	\midrule
	vestibular nuclei   && 前庭核  \\
	
	\midrule
	What/Who pathway/stream  && 内容通路  \\
	
	\midrule
	Where/How pathway/stream && 空间通路  \\
	
	\midrule
	Williams syndrome && 威廉综合症  \\
	
	
	\bottomrule  

\end{longtable}
%}
%\end{table}%


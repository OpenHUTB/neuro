
% 参考: https://zhuanlan.zhihu.com/p/273186198
%\label{chap:preface}
%\begin{table}[htbp]
%	\newcommand{\tabincell}[2]{\begin{tabular}{@{}#1@{}}#2\end{tabular}} %换行指令
%	\centering
%	\caption{名词列表 \label{tab:0_1}}
\renewcommand\arraystretch{1.0}	%设置表格内行间距
%\setlength{\tabcolsep}{4.5mm}{
\begin{longtable}{lll}
% https://blog.csdn.net/maths_girl/article/details/107030167
\caption{名词中英对照表 \label{tab:0_1}} \\
	\toprule 
 中文   && 英文(缩略词) \\
 
 	\midrule
 	abduction     && 外转(角膜向外的眼球运动)   \\
 
	\midrule
	acetylcholine (ACh)     && 乙酰胆碱   \\
	
	\midrule
	adduction     && 内转   \\
	
	\midrule
	agonist–antagonist     &&  兴奋-拮抗  \\
	
	\midrule
	agrin     &&  聚集蛋白  \\
	
	\midrule
	amacrine cell     && 无长突细胞   \\
 
	\midrule
	Best Frequency (BF)     && 最佳频率   \\
 
	\midrule
	Boold Oxygen-Level Dependent (BOLD)     && 血氧水平依赖   \\
	
	\midrule
	choline acetyltransferase (ChAT)   && 胆碱乙酰转移酶  \\
	
	\midrule
	Cochlear Nucleus(CN)   && 耳蜗核  \\
	
	\midrule
	Constant-Frequency (CF)     &&  恒频  \\
	
	\midrule
	cone cell      && 视锥细胞  \\
	
	\midrule
	cortical     &&  皮层  \\
	
	\midrule
	depression, infraduction     &&  下转(角膜向下的眼球运动)  \\
	
	\midrule
	Doppler-shifted constant-frequency (DSCF)     &&  多普勒频移恒频  \\
	
	\midrule
	electrocorticographical (ECoG)     &&  脑电图  \\
	
	\midrule
	elevation, supraduction     &&  上转(角膜向上的眼球运动)  \\
	
	\midrule
	Excitatory PostSynaptic Potential (EPSP)     &&  兴奋性突触后电位  \\
	
	\midrule
	Excited-Excited neuron (EE neuron)     &&  双耳兴奋神经元  \\
	
	\midrule
	Excited-Inhibited neuron (EI neuron)     && 兴奋-抑制神经元   \\
	
	\midrule
	extorsion     && 外旋   \\
	
	\midrule
	formant frequencies     &&  共振峰频率  \\
	
	\midrule
	frequency-modulated (FM)     &&  调频  \\
	
	\midrule
	functional magnetic resonance imaging (fMRI)     &&  功能性磁共振成像  \\
	
	\midrule
	inferior oblique   && 下斜肌  \\
	
	\midrule
	inferior rectus   && 下直肌  \\
	
	\midrule
	Interaural Time Delay(ITD)   && 双耳时间延迟  \\
	
	\midrule
	intorsion   && 内旋  \\
	
	\midrule
	lateral rectus   && 外直肌  \\
	
	\midrule
	Lateral Superior Olivary(LSO)   && 外侧上橄榄  \\
	
	\midrule
	lateral view   && 侧视图  \\
	
	\midrule
	levator   && 眼提肌  \\
		
	\midrule
	Magnetoencephalography (MEG)   && 脑磁图  \\
	
	\midrule
	low-density lipoprotein receptor-related protein 4 (LRP4)   && 低密度脂蛋白受体相关蛋白  \\
	
	\midrule
	Medial Geniculate Body (MGB)   && 内侧膝状体  \\
	
	\midrule
	Medial Nucleus of the Trapezoid Body(MNTB)   && 斜方体内侧核  \\
	
	\midrule
	medial rectus    && 内直肌  \\
	
	\midrule
	Medial Superior 		Olive(MSO)   && 内侧上橄榄  \\
	
	\midrule
	muscle-specific trk-related receptor with a
	kringle domain(MuSK)   && 跨膜受体蛋白酪氨酸激酶  \\
	
	\midrule
	Müllerian inhibiting substance (MIS)   && 缪勒管抑制物  \\
	
	\midrule
	N-Methyl-D-Aspartate (NMDA)   && N-甲基-D-天冬氨酸  \\
	
	\midrule
	optic nerve     && 视神经   \\
	
	\midrule
	photoreceptors     && 光感受器   \\
	
	\midrule
	positron emission tomography (PET)     && 正电子发射断层成像   \\
	
	\midrule
	Posterior Parietal Cortex (PPC)     && 后顶叶皮层   \\
	
	\midrule
	Rapsyn   && 突触受体相关蛋白  \\
	
	\midrule
	primary auditory cortex (A1)   && 初级听觉皮层  \\
	
	\midrule
	Receptive Field (RF)   && 感受野  \\
	
	\midrule
	rectus muscle   && 直肌  \\
	
	\midrule
	rod bipolar   && 杆状双极细胞  \\
	
	\midrule
	rod cell   && 视杆细胞  \\
	
	\midrule
	Rostral auditory cortex (R)   && 嘴侧听觉皮层  \\
	
	\midrule
	Rostrotemporal auditory cortex (R)   && 前颞听觉皮层 \\
	
	\midrule
	Schwann cell   && 施旺细胞 \\
	
	\midrule
	sex-determining region on Y (SRY)   && Y染色体性别决定区 \\
	
	\midrule
	superior oblique   && 上斜肌 \\
	
	\midrule
	superior rectus   && 上直肌 \\
	
	\midrule
	superior view   && 俯视图 \\
	
	\midrule
	tonotopic map   && 音调拓扑图  \\
	
	\midrule
	transverse temporal gyri (Heschl's gyrus)   && 颞横回  \\
	
	\midrule
	V1   && 初级视觉皮层  \\
	
	\midrule
	Ventral Nucleus of the Trapezoid Body(MNTB)   && 斜方体腹侧核  \\
	
	\midrule
	vergence movement   && 聚散运动  \\
	
	\midrule
	version movement   && 同向运动  \\
	
	\midrule
	What/Who pathway/stream  && 内容通路  \\
	
	\midrule
	Where/How pathway/stream && 空间通路  \\
	
	
	\bottomrule  

\end{longtable}
%}
%\end{table}%


\chapter{运动单元和肌肉动作}
% PDF所在目录: /data2/whd/win10/learn/neuro/neuro_神经科学原理_28_中枢神经系统的听觉处理.pdf

\section{运动单元是运动控制的基本单元}
\subsection{一个运动单位由一个运动神经元和多条肌肉纤维组成}
\subsection{运动单元的属性各不相同}
\subsection{身体活动可以改变运动单元的特性}
\subsection{肌肉力量受运动单位募集和放电率的控制}
\subsection{来自脑干的输入改变运动神经元的输入-输出特性}

\section{肌肉力量取决于肌肉的结构}
\subsection{肌节是收缩蛋白的基本组织单位}
\subsection{不可收缩的元素提供必要的结构支撑}
\subsection{收缩力取决于肌纤维激活、长度和速度}
\subsection{肌肉扭矩取决于肌肉骨骼几何结构}

\section{不同的动作需要不同的激活策略}
\subsection{收缩速度可以在大小和方向上变化}
\subsection{肌肉工作取决于激活模式}

\section{要点}
\subsection{选读}
\subsection{参考文献}
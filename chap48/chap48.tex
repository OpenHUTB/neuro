\chapter{突触的形成和消除}


\section{神经元识别特定的突触目标}
\subsection{识别分子促进视觉系统中选择性突触的形成}
\subsection{感觉受体促进嗅觉神经元的靶向}
\subsection{不同的突触输入被定向到突触后细胞的离散域}
\subsection{神经活动增强突触特异性}

\section{神经肌肉接头处揭示了突触分化的原理}
\subsection{运动神经末梢的分化是由肌纤维组织的}
\subsection{突触后肌肉膜的分化是由运动神经组织的}
\subsection{神经调节乙酰胆碱受体基因的转录}
\subsection{神经肌肉接头在一系列步骤中成熟}

\section{中枢突触和神经肌肉接头以相似的方式发育}
\subsection{神经递质受体定位于中央突触}

\section{一些突触在出生后就消失了}

\section{神经胶质细胞调节突触的形成和消除}

\section{要点}
\subsection{选读}
\subsection{参考文献}
\chapter{触觉} \label{chap:chap19}
在关于触觉的这一章中,我们将重点放在手上,因为它对这种方式很重要,特别是它在物体特性的欣赏和熟练的运动任务执行中的作用。 人手是进化的伟大创造之一。 我们手指的精细操作能力之所以成为可能,是因为它们具有精细的感觉能力; 如果我们失去了手指的触觉,我们就失去了手的灵巧性。

无毛皮肤的柔软度和顺应性在触觉中起着重要作用。 当物体接触手时,皮肤会顺应其轮廓,形成物体表面的镜像。 由此产生的皮肤位移和压痕会拉伸组织,从而刺激接触区域或接触区域附近的机械感受器的感觉末梢。

当我们用手操纵物体和探索世界时,这些感受器高度敏感并持续活跃。 它们向大脑提供有关物体在手中的位置、形状和表面纹理、施加在接触点的力的大小以及手或物体移动时这些特征如何随时间变化的信息。 指尖是身体中神经支配最密集的部位之一,提供了关于手部操作的物体的广泛而冗余的体感信息。

此外,手的解剖结构及其多个关节和适当的手指使人类能够以反映物体整体形状的方式塑造手,从而提供以手为中心的外部世界本体感受表征。 这种内化物体形状的能力使我们能够创造出可以单独扩展我们双手能力的工具。

当我们熟练使用手术刀或剪刀等工具时,我们会感觉到工具工作表面的状况,就好像我们的手指在那里一样,因为两组触觉感受器监测工具产生的振动和力 那些遥远的条件。 当我们用手指扫过一个表面时,我们会感觉到它的形状和质地,因为另一组机械感受器具有很高的空间和时间敏锐度。 盲人利用这种能力以每分钟一百个单词的速度阅读盲文。 当我们抓握和操纵一个物体时,我们会非常小心地使用所需的力量,因为特定的机械感受器会持续监控滑动并适当地调整我们的抓握。

我们还能够仅通过触摸来识别放在手中的物体。 当我们拿到棒球时,由于它的形状、大小、重量、密度和质地,我们无需看就能立即认出它。 我们不必考虑每个手指提供的信息就可以推断出该物体一定是棒球; 信息流入内存并立即匹配先前存储的棒球表示。 即使我们以前从未接触过棒球,我们也会将其视为单个物体,而不是离散特征的集合。 大脑的体感通路有一项艰巨的任务,即整合来自每只手上数千个传感器的信息,并将其转化为适合认知和行动的形式。

为运动控制和认知目的提取感官信息,并为这些目的提取不同种类的信息。 例如,我们可以将注意力从棒球的形状转移到它在手中的位置,以重新调整我们的抓地力以进行有效的投掷或投球。 这种对感觉信息不同方面的选择性关注是由皮质机制引起的。

\section{主动和被动触摸有不同的目标}
触摸被定义为两个身体之间的直接接触。 在神经科学中,触觉是指有意识地感知与身体接触的特殊感觉。 触摸可以是主动的,例如当您将手或身体的其他部分移动到另一个表面时,也可以是被动的,例如当某人或其他东西触摸您时。 主动接触从根本上说是一个自上而下的过程,在这个过程中,主体有代理权,寻求特定信息,并控制发生的事情。 主体选择对象的相关显着特征来确定后续行为。 他们选择要抓住的物体和获得它所需的最有效的手形,并决定如何操纵它以实现特定目标。 在主动触摸过程中,体感信息描述了物体的物理特性以及受试者手和手臂的运动,以及它们与任务目标的关系。 重要的是,物体的主动操作是基于触摸的概念,作为一种三维模态,旨在捕捉物体的体积、地形和弹性特性,正如 Roberta Klatzky 和 Susan Lederman 首次提出的那样。 这些三维品质最好通过主动操作来欣赏,包括用手抓握、旋转和轮廓追踪。

被动触摸采用自下而上的过程,在该过程中,受试者对实验者或临床医生指定的外部刺激做出反应。 实验者选择并控制传递到皮肤的刺激的位置、幅度、力、时间、持续时间和空间分布。 随后的行为由范例中提供的指令指导。 触觉刺激分为实验者选择的类别和/或按照密集或享乐等级进行评级。 因此,受试者需要分析所有传输的体感信息,并在部分任务说明的指导下选择特定特征。

触觉刺激的主动和被动模式会激发皮肤中相同的受体群,并在传入纤维中引起类似的反应。 它们在反映刺激期间注意力和行为目标的认知特征上有所不同。 通过命名对象或描述感觉来测试被动触觉; 当手操纵物体时使用主动触摸。 触觉的感觉和运动成分在大脑中在解剖学上密切相关,并且在指导运动行为方面具有重要的功能。

在主动触摸过程中,来自大脑皮层运动中枢的下行纤维终止于内侧背角的中间神经元,该中间神经元从皮肤接收触觉信息。 来自皮层运动区的类似纤维终止于背柱核,提供产生行为的运动指令的效应副本(或推论放电)(第 30 章)。 以这种方式,来自手的主动手部运动产生的触觉信号可以集中地与神经学检查或心理物理学测试中被动施加的刺激区分开来。

当患者手部使用有缺陷时,主动和被动触摸之间的区别在临床上很重要。 感觉丧失可能导致无力、僵硬或笨拙等运动缺陷,这就是为什么被动感觉测试在神经系统检查中很重要。 常见的触觉神经学测试包括检测阈值、振动感、两点或纹理辨别力,以及通过触觉识别形状的能力(立体视觉)。 这些测试测量各种触觉感受器的灵敏度和功能。 与预期值的偏差可能有助于诊断感觉缺陷或导致体感功能障碍的病变。 本章讨论了这些测试背后的神经机制。 其他常见的体感功能测试——腱反射、针刺和热测试——在其他章节中讨论。


\section{手有四种类型的机械感受器}
人手的触觉感觉来自四种机械感受器:Meissner 小体、Merkel 细胞、Pacinian 小体和 Ruffini 末梢(图 19-1)。 每个受体根据其形态、神经支配模式和皮肤深度以独特的方式做出反应。 触觉可以理解为这四个系统协同作用所提供信息的综合结果。

触觉感受器由缓慢适应或快速适应的轴突支配。 缓慢适应 (SA) 纤维通过持续放电响应稳定的皮肤压痕,而快速适应 (RA) 纤维在压痕变得静止时停止放电(图 19-1 和表 19-1)。 手部持续的机械感觉必须相应地从 SA 纤维中产生; 皮肤上或穿过皮肤的运动感觉主要由 RA 纤维发出信号。

手上的触觉感受器根据大小和在皮肤中的位置进一步细分为两种类型。 1 型触觉纤维终止于真皮和表皮之间边缘皮肤表层的小受体器官簇(迈斯纳小体或默克尔细胞)(图 19-2,方框 19-1.)。 RA1 纤维是手部数量最多的触觉传入神经,在人和猴子的指尖处达到约 150 个/cm2 的密度; SA1 纤维也广泛分布在手中,指尖的密度为每平方厘米 70 根。

2 型纤维稀疏地支配皮肤并终止于位于真皮或皮下组织中的单个大受体(Pacinian 小体和 Ruffini 末梢)。 这些受体比 1 型纤维的受体器官更大但数量更少。 2 型受体的大尺寸使它们能够在距感觉神经末梢一定距离处感知皮肤的机械位移。 人手指中RA2纤维的密度仅为每平方厘米21根; SA2 纤维含量最少,每平方厘米仅提供 9 根纤维。

\begin{figure}[htbp]
	\centering
	\includegraphics[width=0.5\linewidth]{chap19/fig_19_3}
	\caption{人类指尖的皮肤。 
		A. 人类食指指纹的扫描电子显微照片。 
		手部的无毛皮肤由乳头状脊和间隔有规律的沟(限制脊)排列而成。 
		汗珠从乳头状嵴中心的导管中渗出,沿着每个隆起的中心形成规则间隔的网格状图案。 
		Merkel 细胞位于表皮底部汗管下方的密集簇中,沿着乳头状脊的中心(见图 19-2)。
		 (经许可改编自 Quilliam 1978。) 
		 B. 平行于皮肤表面切割的无毛皮肤的组织切片。 此处对胆碱酯酶进行免疫染色的迈斯纳小体沿着与限制脊相邻的每个乳头状脊的两侧形成规则间隔的链。 
		 因此,Meissner 小体和 Merkel 细胞形成交替的快速适应 1 型 (RA1) 和缓慢适应 1 型 (SA1) 触摸感受器带,跨越每个指纹脊。 (经许可改编自 Bolanowski 和 Pawson 2003。)}
	\label{fig:19_3}
\end{figure}


\subsection{细胞的感受野决定了它的触觉敏感区}
单个机械感受器纤维从称为感受野的有限皮肤区域传递信息(第 18 章)。 Åke Vallbo 和 Roland Johansson 首先使用显微神经描记术研究了人手的触觉感受野。 他们通过皮肤将微电极插入人类前肢的正中神经或尺神经,并记录单个传入纤维的反应。 他们发现,与其他灵长类动物一样,人类的触觉感受器在生理反应和感受野结构方面都存在重要差异。

1 型纤维具有小的、高度局部化的感受野,具有多个高灵敏度点,反映了它们在皮肤中的轴突末端的分支模式(图 19-5)。 一个 RA1 轴突通常支配 10 到 20 个迈斯纳小体,整合来自几个相邻指纹脊的信息。 一根 SA1 纤维支配年轻人中大约 20 个 Merkel 细胞(图 19-4B); 随着年龄的增长,默克尔细胞的数量会显着下降。

相比之下,支配皮肤深层的 2 型纤维仅与单个 Pacinian 小体或 Ruffini 末梢相连。 由于这些受体很大,它们从更广泛的皮肤区域收集信息。 他们的感受野通常包含一个单一的“热点”,那里对触摸的敏感度最高; 该点位于接收器的正上方(图 19-5)。

指尖的感受野在身体上是最小的,SA1 纤维平均为 11 mm2,RA1 纤维平均为 25 mm2。 小场补充了指尖中高密度的感受器。 近端指骨和手掌的感受野逐渐变大,这与这些区域的机械感受器密度较低一致。 重要的是,1 型纤维的感受野明显小于大多数接触手的物体,因此仅表示物体有限部分的空间特性。 与视觉系统一样,物体的空间特征分布在受刺激的受体群中,这些受体的反应在大脑中整合形成统一的感知。

每个 RA2 轴突在没有分支的情况下终止于单个 Pacinian 小体,并且每个 Pacinian 小体只接受一个 RA2 轴突。 Pacinian 小体是大的洋葱状结构,其中连续的结缔组织层被充满液体的空间隔开(见图 19-8A1)。 这些层围绕着无髓鞘的 RA2 末端及其有髓鞘的轴突,直至一个或多个朗飞节。 胶囊会放大高频振动,这一作用对于工具的使用非常重要。 据估计,人手中的帕西尼氏小体数量从年轻人的 2,400 个到老年人的 300 个不等。

SA2 纤维支配 Ruffini 末梢,集中在手指和腕关节、指甲周围的皮肤以及手掌的皮肤褶皱处。 Ruffini 末端是细长的纺锤形结构,包围着从皮下组织延伸到关节、手掌或指甲边缘处皮肤褶皱处的胶原纤维。 SA2 神经末梢缠绕在囊中的胶原纤维之间,就像在高尔基肌腱器官中一样(方框 32-4),并被沿其长轴拉伸皮肤的刺激所兴奋。


\subsection{两点辨别测试测量触觉敏锐度}
人类分辨纹理表面空间细节的能力取决于接触身体的哪个区域。 当一对探针在手上相距几毫米时,每个探针都被视为一个不同的点,因为它在皮肤上产生一个单独的酒窝并刺激不重叠的受体群。 当探头靠得更近时,两种感觉会变得模糊,因为两个探头都包含在相同的感受野中。 触觉刺激之间的空间相互作用构成了两点辨别和纹理识别的神经学测试的基础。

触觉敏锐度的阈值——定义机会和完美辨别之间性能的分离——在年轻人的指尖上约为 1 毫米,但在老年人中下降到约 2 毫米。 指尖和嘴唇的触觉敏锐度最高,感受野最小。 身体近端部位的触觉敏锐度随着 SA1 和 RA1 纤维感受野的大小而降低(图 19-6A)。

当我们抓住或触摸一个物体时,我们可以辨别其表面的特征,相隔小至 0.5 毫米。 人类能够区分脊间距非常窄的光栅的水平和垂直方向(图 19-6B)。 长边,如光栅的脊,在同时刺激感受野中的多个感觉末梢时会引起 RA1 和 SA1 传入神经的更强反应,强调多传感器感受野对触觉信息处理的重要性。 Roland Johansson 和 Andrew Pruszynski 最近发现,RA1 和 SA1 纤维对接触多个感觉末梢的边缘反应更强烈,使这些传入神经能够区分垂直、水平或倾斜方向。

女性的触觉敏锐度略高于男性,并且在手指之间存在差异,但在手之间没有差异; 性别差异主要与女性乳头嵴直径较小以及由此导致的每平方厘米皮肤 SA1 纤维密度较高有关。 食指的远端垫具有最敏锐的灵敏度; 空间敏锐度从食指到小指逐渐下降,并在靠近远端指垫的位置迅速下降。 小指远端的触觉空间分辨率较差 50\%,手掌较粗糙 6 到 8 倍。

盲人使用 SA1 和 RA1 纤维的精细空间灵敏度来阅读盲文。 盲文字母表将字母表示为易于通过触摸区分的简单圆点图案。 盲人通过在圆点图案上移动手指来阅读盲文。 这种手部运动增强了点产生的感觉。 由于盲文点之间的间距约为 3 毫米,该距离大于 SA1 纤维的感受野直径,因此每个点刺激一组不同的 SA1 纤维。 当一个点进入它的感受野时,SA1 纤维会发出一阵动作电位,而一旦该点离开感受野,它就会停止活动(图 19-7)。 同步发射的 SA1 纤维的特定组合发出盲文点的空间排列信号。 RA1 纤维还区分点图案,增强 SA1 纤维提供的信号。

尽管 Pacinian 小体(RA2 纤维)对扫描皮肤上的盲文点有反应,但它们的尖峰序列并不反映盲文图案中点的周期性。 相反,它们会发出由盲文点在皮肤上的运动引起的皮肤振动信号。 Sliman Bensmaia 及其同事最近发现,当用这种方法测试织物等精细质地时,RA2 传入信号通过生成与这些表面特征同相的尖峰序列来发出编织物中线的周期性信号。 SA1 纤维对纺织品运动的反应较差,因为螺纹尺寸通常太小,无法以足够的幅度压入皮肤。 尽管如此,所有三种类型的触觉传入都有助于人类感知粗糙度和光滑度。


\begin{figure}[htbp]
	\centering
	\includegraphics[width=0.5\linewidth]{chap19/fig_19_8}
	\caption{快速适应型 2 (RA2) 纤维具有最低的振动阈值。 
		振动是由皮肤的正弦波刺激产生的感觉,如电动机的嗡嗡声、乐器的弦或神经系统检查中使用的音叉。 
		A. 1. Pacinian 小体由包裹 RA2 纤维末端的同心、充满液体的结缔组织薄片组成。 这种结构特别适合运动检测。 
		RA2 纤维中的感觉转导发生在与胶囊内层相连的拉伸敏感阳离子通道中。 
		2. 当对皮肤施加稳定的压力时,RA2 纤维会在刺激开始和结束时爆发。 
		作为对正弦刺激(振动)的响应,纤维会定期放电,这样每个动作电位都会发出一个刺激周期的信号。 
		我们将振动视为有节奏地重复的事件,是由于同时激活许多 RA2 单元而产生的,这些单元同步发射。 (改编自 Talbot et al. 1968。) 
		B. 1. 振动检测的心理物理阈值取决于刺激频率。 
		如图所示,人类在抓取大物体时可以检测到 200 Hz 时小至 30 nm 的振动; 阈值在其他频率和使用小探头测试时更高。 (经许可改编自 Brisben、Hsiao 和 Johnson 1999。) 
		2. 人类的振动阈值,通过压入皮肤的小探针尖端测量,与每个频率范围内最敏感的触摸纤维的阈值相匹配。 
		每种类型的机械感觉纤维对特定频率范围最敏感。 
		慢速适应 1 型 (SA1) 纤维是 5 赫兹以下最敏感的群体,快速适应 1 型 (RA1) 纤维在 10 赫兹和 50 赫兹之间,RA2 纤维在 50 赫兹和 400 赫兹以上。 (经许可改编自 Mountcastle、LaMotte 和 Carli 1972,以及 Johansson、Landström 和 Lundström 1982。)}
	\label{fig:19_8}
\end{figure}



\subsection{缓慢适应的纤维检测物体压力和形状}

SA1 和 SA2 纤维最重要的功能是它们能够发出皮肤变形和压力信号。 SA1 受体对边缘、拐角、点和曲率的敏感性提供了有关物体顺应性、形状、大小和表面纹理的信息。 如果当我们触摸一个物体时它会压进皮肤,我们会认为它是硬的或坚硬的,如果我们使它变形,我们会认为它是软的。

矛盾的是,随着物体尺寸和直径的增加,其表面曲率会降低。 单个 SA1 纤维的反应较弱,由此产生的感觉感觉不那么明显。 例如,将铅笔尖压入皮肤 1 毫米,感觉尖锐、令人不快,并且在接触点处高度局限,而橡皮擦的 1 毫米压痕感觉又钝又宽。 最弱的感觉是由压在指垫上的平面引起的。

要理解为什么这些物体会引起不同的感觉,我们需要考虑触摸皮肤时发生的物理事件。 当铅笔尖压在皮肤上时,它会在接触点处使表面凹陷,并在周围区域(半径约 4 毫米)形成一个浅而倾斜的盆地。 虽然压痕力集中在中心,但周围区域也受到局部拉伸的扰动,称为拉伸应变。 位于皮肤中心和周围“山坡”的 SA1 受体受到刺激,发射与局部拉伸程度成比例的尖峰序列。

如果将第二个探针压近第一个探针,则会刺激更多的 SA1 纤维,但每根纤维的神经反应都会降低,因为移动皮肤所需的力由两个探针共享。 Ken Johnson 和他的同事已经表明,随着在感受野中添加更多的探针,每个感觉末端的反应强度会逐渐变弱,因为皮肤上的位移力分布在整个接触区。 因此,皮肤力学导致“少即是多”的情况。 单个 SA1 纤维对小物体的反应比对大物体的反应更强烈,因为压入皮肤所需的力集中在一个小接触点。 以这种方式,每个 SA1 纤维都将局部皮肤压痕轮廓整合到其接受域内。

SA1 受体对皮肤局部应变的敏感性使它们能够检测到边缘,即物体曲率突然变化的地方。 当手指触摸边缘时,SA1 的发射率比触摸平面时高许多倍,因为物体边界施加的力使皮肤不对称地移位,超出边缘以及在边缘处。 这种力的不对称分布增强了位于物体边缘的感受野的响应。 由于边缘通常被认为是锋利的,我们倾向于在平坦或轻微弯曲的表面上而不是通过边缘来抓住物体。

支配 Ruffini 末端的 SA2 纤维对皮肤拉伸的反应比对压痕的反应更强烈,因为它们的解剖学位置沿着手掌皱襞或手指关节。 它们提供有关用整只手抓住大物体形状的信息,即物体被压在手掌上的“力量抓握”。

SA2 系统可能在立体视觉(仅使用触摸来识别三维物体)以及其他以皮肤拉伸为主要线索的感知任务中发挥核心作用。 Benoni Edin 表明,手背毛茸茸的皮肤的 SA2 神经支配在手形和手指位置的感知中起着重要作用。 SA2 纤维通过检测指关节或手指之间的织带中的皮肤拉伸来帮助感知手指关节角度。 这些关节附近的 Ruffini 末端对齐,以便在手指向特定方向移动时刺激不同的感受器组(图 19-5A,底部面板)。 以这种方式,SA2 系统提供了整个手部皮肤伸展的神经表征,一种本体感受而非外感受功能。

当手空着时,SA2 纤维还提供有关手形和手指运动的本体感受信息。 如果手指完全伸展和外展,我们会感觉到手掌和近端指骨的拉伸,因为无毛的皮肤变平了。 同样,如果手指完全弯曲并握拳,我们会感觉到手背皮肤的拉伸,尤其是掌指关节和近端指间关节。 人类使用这种本体感受信息来预先塑造他们的手以有效地抓住物体,将手指张开到足以清除物体并熟练地抓住它而不用太大力。

\subsection{快速适应纤维检测运动和振动}
振动觉测试是神经学检查的重要组成部分。 用以特定频率振荡的音叉触摸皮肤会引起周期性的嗡嗡声,因为大多数触觉感受器会同步发射与刺激频率同相的周期性动作电位序列(图 19-8A2)。 振动感是对触摸的动态敏感度的有用测量,特别是在局部神经损伤的情况下。

RA2 受体,即 Pacinian 小体,是体感系统中最敏感的机械感受器。 它对高频(30-500 赫兹)振动刺激反应灵敏,可以检测纳米范围内 250 赫兹的振动(图 19-8B2)。 Pacinian 小体过滤和放大高频振动的能力使我们能够感觉到手中工具工作表面的状况,就好像我们的手指本身正在触摸工具下的物体一样。 临床医生利用这种敏锐的灵敏度将针头引导到血管中并探测组织硬度。 汽车修理工使用振动感将扳手定位在看不见的螺栓上。 我们可以在黑暗中书写,因为我们能感觉到钢笔接触纸张时的振动,并将摩擦力从表面粗糙度传递到我们的手指。

尽管 Pacinian 小体对于大于 40 Hz 的频率具有最低的振动阈值(图 19-8B2),更高振幅的振动刺激也会激发 SA1 和 RA1 纤维,即使它们诱发的尖峰序列比 Pacinian 传入神经弱。 图 19-9A 说明了 15 种不同周围神经纤维在 20 Hz 下以弱、中等和高振幅刺激的诱发放电模式。 尽管这些纤维对振动的敏感度不同,但它们的尖峰序列具有某些重要的共同特征。 首先,每个神经元在振动周期的特定阶段放电,通常是当探头压入皮肤时,其尖峰相位模式复制振动频率:当以 20 赫兹刺激时,尖峰脉冲以大约 50 毫秒的间隔重复出现。 尖峰序列的模式得到进一步加强,因为纤维群同步发射,使频率信息能够由于突触整合而集中保存。

随着刺激幅度的增加,每次爆发的尖峰总数也增加,允许每根纤维多路复用振动频率和强度的信号:频率信息由尖峰序列的时间模式传达,振动幅度由总编码 每根纤维每秒发射的尖峰数,以及激活纤维集合的总尖峰输出。 最后,请注意,每个神经元的尖峰序列在时间过程中非常相似,并且每个条件下每次试验的尖峰计数都非常相似,这表明触觉传入纤维提供的感觉信号具有很高的可靠性。 感觉编码的这种可靠性和可预测性使振动成为评估触觉的一种特别有用的技术。

\subsection{缓慢和快速适应的纤维对握力控制都很重要}
除了感知物体的物理特性外,触觉感受器还提供有关熟练动作中手部动作的重要信息。 Roland Johansson 和 Gören Westling 使用显微神经造影术来确定当物体被抓在手中时触觉感受器的作用。 通过将微电极放置在正中神经中,他们能够记录触摸纤维在手指最初接触物体时的放电模式,以及当物体被拇指和食指抓住、举起、放在桌子上方、放低、 并返回休息。

他们发现所有四类触摸纤维都对抓握有反应,并且每种纤维类型都监控特定功能。 在没有触觉刺激的情况下,RA1、RA2 和 SA1 纤维通常是沉默的。 当物体第一次被触摸时,它们会检测到接触(图 19-10)。 SA1 纤维发出每个手指施加的握力大小的信号,RA1 纤维感知施加抓握的速度。 RA2 纤维检测从桌子上提起和返回时通过物体传输的小冲击波。 我们知道物体何时由于这些振动而与桌面接触,因此可以在不看物体的情况下操纵物体。 RA1 和 RA2 纤维在掌握建立后停止响应。 SA2 纤维在抓取或释放物体期间发出手指弯曲或伸展的信号,从而在这些运动进行时监测手部姿势。

来自手的报告物体形状、大小和质地的信号是控制抓取过程中力应用的重要因素。 Johansson 和他的同事发现,我们举起和操纵物体时非常灵敏——握力刚好超过导致明显滑动的力——并且握力会自动调整以补偿手指和物体之间摩擦系数的差异 表面。 受试者预测抓住和举起物体需要多大的力,并根据 SA1 和 RA1 传入神经提供的触觉信息修改这些力。 具有光滑表面的物体比具有粗糙纹理的物体更牢固地抓住,在手与物体的初始接触期间由 RA1 传入编码的属性。 在神经损伤或手部局部麻醉期间,可以看到触觉信息在抓握中的重要性; 患者施加异常高的握力,手指施加的握力和负载力之间的协调性很差。

RA1 受体提供的用于监测抓握动作的信息对于抓握控制至关重要,它使我们能够在扰动导致物体意外滑落时抓住物体。 RA1 纤维在稳定抓握过程中保持安静,并且通常保持安静,直到物体恢复静止并释放抓握。 然而,如果物体出乎意料地重或被外力摇晃并开始从手中滑落,RA1 纤维会响应物体的小切向滑动运动而激发。 RA1 活动的最终结果是来自运动皮层的信号增加了握力。


\section{触觉信息在中央触摸系统中处理}
支配手的感觉传入纤维通过正中神经、尺神经和浅表桡神经将触觉和其他体感信息传递到中枢神经系统。 这些神经同侧终止于 C6 至 T1 脊柱节段; 这些纤维的其他分支通过同侧背柱直接投射到髓质,在那里它们与楔形核中的神经元建立突触连接,楔形核是背柱核的外侧分裂(图 19-11)。

\subsection{脊髓、脑干和丘脑回路分离触觉和本体感觉}
背柱中的纤维和背柱核中的神经元按地形图组织,上半身(包括手)在楔形束和核的外侧表示,下半身在细束和核的内侧表示。 触觉和本体感觉的体感子模态在这些区域中也在功能上分离,因为单个脊髓和脑干神经元从单一类型的传入神经元接收突触输入,而不同类型的神经元在空间上是分离的。 背柱核的头侧三分之一由处理来自肌肉传入神经的本体感受信息的神经元控制; 触觉输入在尾部更占主导地位。 模态隔离是投射到初级体感皮层的通路的一贯特征。

背柱核中的神经元将其轴突投射到延髓的中线,形成内侧丘系,这是一种突出的纤维束,可将触觉和本体感受信息从身体的对侧通过脑桥和中脑传递到丘脑。 由于感觉纤维的这种交叉(或交叉),大脑的左侧从身体右侧的机械感受器接收体感输入,反之亦然。 在运输过程中,身体在内侧丘系和丘脑内的躯体表征变得颠倒; 身体的地形图在中间显示面部,在侧面显示下半身,在中间显示上半身和手。

来自手和身体其他区域的触觉和本体感受信息在丘脑的不同亚核中进行处理。 来自四肢和躯干的触摸信号通过内侧丘系被发送到腹侧后外侧 (VPL) 核,而来自面部和嘴巴的触摸信号被传送到腹侧后内侧 (VPM) 核。 来自肌肉和关节(包括手部)的本体感受信息被传输到腹侧后上核 (VPS)。 这些核将它们的输出发送到大脑皮层顶叶的不同子区域。 VPL 和 VPM 核主要将皮肤信息传递到初级体感皮层 (S-I) 的 3b 区,而 VPS 核主要将本体感受信息传递到 3a 区。

\subsection{体感皮层被组织成功能专门的列}
有意识的触觉意识被认为起源于大脑皮层。 触觉信息通过位于顶叶中央后回的初级体感皮层 (S-I) 进入大脑皮层。 初级躯体感觉皮层包括四个细胞结构区域:布罗德曼区域 3a、3b、1 和 2(图 19-12)。 这些区域相互关联,因此 S-I 中的感官信息处理涉及串行和并行处理。

在对大脑皮层的一系列开创性研究中,Vernon Mountcastle 发现 S-I 皮层被组织成垂直的柱状或板状结构。 每列宽 300 至 600 μm,横跨从软脑膜表面到白质的所有六个皮质层(图 19-13)。 列内的神经元接收来自同一局部皮肤区域的输入,并对同一类或多类触觉感受器做出反应。 因此,纵柱包含新皮质的基本功能模块; 它提供了一个解剖结构,组织感官输入以传达有关位置和形态的相关信息。

皮质的柱状组织是内在皮质回路、丘脑皮质轴突的投射模式和皮质发育过程中成神经细胞迁移途径的直接结果。 列内的连接模式垂直定向,垂直于皮质表面。 丘脑皮质轴突主要终止于第 IV 层的星状细胞簇,其轴突垂直投射到皮质表面,以及星形金字塔细胞。 因此,丘脑皮质输入被中继到锥体细胞的窄列,这些细胞与第 IV 层细胞轴突接触。 其他皮质层中皮质锥体细胞的顶端树突和轴突也主要垂直定向,平行于丘脑皮质轴突和星状细胞轴突(图 19-14)。 这使得相同的信息可以由整个皮层厚度的一列神经元处理。

锥体神经元构成躯体感觉皮层的主要兴奋类; 它们构成了大约 80\% 的 S-I 神经元。 六个皮质层中每一层的锥体神经元投射到特定目标(图 19-14)。 循环水平连接连接相同或相邻列中的锥体神经元,允许它们在被相同刺激同时激活时共享信息。 II 层和 III 层中的神经元也投射到同一列中的 V 层,投射到同一半球的更高皮质区域,并投射到相反半球的镜像位置。 这些与更高皮层区域的前馈连接允许复杂的信号整合,如本章后面所述。

V 层中的锥体神经元提供每一列的主要输出。 它们接收来自同一列和相邻列中 II 层和 III 层神经元的兴奋性输入以及稀疏的丘脑皮层输入。 V 层(V-A 层)表层部分的神经元将前馈输出双向发送到高阶皮层区域的 IV 层(参见图 19-17C)以及纹状体。 V 层(V-B 层)更深的神经元投射到皮层下结构,包括基底神经节、上丘、桥脑和其他脑干核团、脊髓和背柱核团。 第六层神经元投射到局部皮层神经元,然后返回丘脑,特别是投射到为该柱提供输入的腹侧后核区域。

除了来自触觉感受器的信息前馈信号外,来自较高体感皮层区域的第 II 层和第 III 层的反馈信号被提供给较低皮层区域的第 I 层,从而调节它们的兴奋性。 这种反馈信号不仅起源于躯体感觉皮层区域,还起源于后顶叶皮层的感觉运动区域、额叶运动区域、边缘区域以及参与记忆形成和存储的内侧颞叶区域。 这些反馈信号被认为在选择用于认知处理(通过注意力机制)和短期记忆任务的感觉信息方面发挥作用。 反馈通路也可能在运动活动期间控制感觉信号。 每列内的各种局部抑制性中间神经元用于聚焦柱状输出。

\subsection{皮层柱是按体位组织的}
初级躯体感觉皮层内的列按地形排列,因此在 S-I 的四个区域中的每一个区域都有一个完整的躯体表示(图 19-15)。 身体的皮层图大致对应于脊柱皮区(见图 18-13)。 骶骨节段在内侧,腰椎和胸椎节段在中央,颈椎节段在外侧,面部的三叉神经节段在 S-I 皮质的最外侧部分。 了解大脑中身体的神经图对于定位中风或头部外伤对皮质的损伤非常重要。

体表在顶叶中至少有 10 个不同的神经映射:四个在 S-I 中,四个在 S-II 中,至少两个在后顶叶皮层中。 因此,这些区域调节触觉的不同方面。 S-I 区域 3b 和 1 中的神经元处理表面纹理的细节,而区域 2 中的神经元代表物体的大小和形状。 躯体感觉的这些属性在 S-II 和后顶叶皮层中得到进一步阐述,神经元分别参与物体辨别和操纵。

体位图的另一个重要特征是分配给每个身体部位的大脑皮层数量。 人脑中身体的神经图,称为矮人,并不完全复制皮肤的空间地形。 相反,身体的每个部分都按照其对触觉的重要性来表示。 不成比例的大面积用于某些身体区域,特别是手、足和嘴,而相对较小的区域用于更近端的身体部位。 在人类和猴子中,更多的皮质柱用于手指而不是整个躯干(图 19-15C)。

用于单位面积皮肤的皮层面积的数量——称为皮层放大率——在不同的身体表面上相差超过一百倍。 它与神经支配密度密切相关,因此与皮肤区域中触觉感受器的空间敏锐度密切相关。 人类大脑中放大倍数最大的区域——嘴唇、舌头、手指和脚趾——的触觉敏锐度阈值分别为 0.5、0.6、1.0 和 4.5 毫米。

啮齿动物和其他用胡须探测环境的哺乳动物在 S-I 中有大量列,称为桶,接收来自面部单个触须的输入(方框 19-2)。 桶状皮质为研究皮质电路提供了广泛使用的实验准备。

\subsection{皮层神经元的感受野整合来自邻近受体的信息}
S-I 中的神经元至少是皮肤中触觉感受器之外的三个突触。 他们的输入代表在背柱核、丘脑和皮层本身处理的信息。 每个皮层神经元接收来自皮肤特定区域的受体的输入,这些输入一起构成它的感受野。 我们认为皮肤上的特定位置被触摸是因为皮层中特定的神经元群被激活。 这种体验可以通过对相同皮层神经元的电刺激或光遗传学刺激进行实验诱导。

皮层神经元的感受野比周围神经中的体感纤维大得多。 例如,支配指尖的 SA1 和 RA1 纤维的感受野是皮肤上的小点(图 19-5),而接收这些输入的皮层神经元的感受野覆盖整个指尖或几个相邻的手指(图 19-17B) . 区域 3b 中神经元的感受野代表 300 到 400 根神经纤维的输入复合,通常覆盖单个指骨或掌垫。 来自同一皮肤区域的 SA1 和 RA1 触觉感受器的输入会聚到区域 3b 中的共同神经元。

较高皮质区域的感受野甚至更大,跨越在运动活动期间同时激活的皮肤功能区域。 这些包括几个相邻手指的指尖,或整个手指,或手指和手掌。 S-I 区域 1 和 2 中的神经元关注的信息比它们在身体上的神经支配部位更抽象。 当同时触摸多个手指时,其感受野包括一根以上手指的神经元会以更高的速率放电,并以这种方式发出手中物体的大小和形状的信号。 这些大的感受野允许皮层神经元整合来自各个触觉感受器的碎片化信息,使我们能够识别物体的整体形状。 例如,这样的神经元可以区分螺丝刀的手柄和刀刃。

来自 S-I 中不同感觉受体的会聚输入也可能允许单个神经元检测物体的大小和形状。 区域 3b 和 1 中的神经元仅对触摸有反应,而区域 3a 中的神经元对肌肉拉伸有反应,而区域 2 中的许多神经元都接收这两种输入。 因此,区域 2 中的神经元可以整合有关用于抓取物体的手形、手施加的握力以及物体产生的触觉刺激的信息; 这种综合信息可能足以识别物体。

皮层神经元的感受野通常有一个被抑制区包围或叠加的兴奋区(图 19-18A)。 刺激兴奋区外的皮肤区域可能会降低神经元对感受野内触觉刺激的反应。 类似地,感受野内的重复刺激也可能降低神经元反应性,因为该通路的兴奋性因局部中间神经元介导的更持久抑制而减弱。

抑制性感受野是通过背柱核、丘脑和皮层本身中的中间神经元的前馈和反馈连接产生的,它们限制了兴奋的传播。 由一个回路中的强烈活动产生的抑制作用会降低附近仅处于微弱兴奋状态的神经元的输出。 抑制网络确保传递几种竞争反应中最强的一种,从而允许采用赢者通吃的策略。 当大量的触摸神经元受到刺激时,这些电路可以防止纹理等触觉细节模糊。 此外,当手用于熟练任务时,大脑中的高级中枢会使用抑制回路,通过抑制不需要的、分散注意力的输入,将注意力集中在手的相关信息上。

感受野在皮肤上的大小和位置不是永久固定的,而是可以根据经验或感觉神经的损伤进行修改(第 53 章)。 皮质感受野似乎是在发育过程中形成的,并通过同时激活输入通路来维持。 如果周围神经受伤或被横切,其皮质投射目标会从通常被抑制网络抑制的不太有效的感觉输入,或从保留神经支配的邻近皮肤区域新开发的连接中获得新的感受野。 同样,通过重复练习广泛刺激传入通路可能会加强突触输入,从而改善知觉,从而提高表现。


\section{触摸信息在连续的中央突触中变得越来越抽象}
体感信息从 S-I 的四个区域并行传递到皮层的更高中心,例如次级体感皮层 (S-II)、后顶叶皮层和初级运动皮层(图 19-17C)。 随着信息流向更高阶的皮层区域,需要特定的刺激模式组合来激发单个神经元。

来自邻近神经元的信号在更高的皮层区域被组合以辨别物体的全局特性,例如它们在手上的方向或运动方向(图 19-19)。 一般来说,较高皮层区域的皮层神经元关注独立于其感受野中刺激位置的感觉特征,抽象出特定类别刺激共有的对象属性。

由于突触前感受野的空间排列,皮层神经元能够检测边缘的方向或运动方向。 兴奋性突触前神经元的感受野通常沿着一个公共轴对齐,该轴生成突触后神经元的首选方向。 此外,位于兴奋区一侧的抑制性突触前神经元的感受野增强了突触后神经元的定向和方向选择性(图 19-18B)。

\subsection{认知触觉由次级躯体感觉皮层中的神经元介导}
S-I 神经元对触摸的反应主要取决于神经元感受野内的输入。 这种前馈通路通常被描述为自下而上的过程,因为外围的受体是 S-I 皮层神经元兴奋的主要来源。

高阶体感区域不仅从外周受体接收信息,而且还受到自上而下的认知过程的强烈影响,例如目标设定和注意力调节。 从各种研究中获得的数据——猴子的单神经元研究、人类神经影像学研究以及高阶体感区域受损患者的临床观察——表明顶叶的腹侧和背侧区域在触觉系统中起着互补的作用 类似于视觉系统的“什么”和“哪里”通路(见图 17-13)。

S-II 位于人类和猴子外侧沟的上岸和相邻的顶叶盖(图 19-12B 和 19-20B)。 与 S-I 一样,S-II 皮层包含四个不同的解剖学子区域,具有单独的身体图。 中央区域——由 S-II 本身和相邻的顶叶腹侧区域组成——接收来自区域 3b 和 1 的主要输入,主要是来自手和面部的触觉信息。 一个更靠近头端的区域,即顶叶腹侧区域,从区域 3a 接收有关主动手部运动的信息以及来自区域 3b 和 1 的触觉信息(图 19-20)。 外侧沟最尾端的体感区域延伸到顶叶盖(图 19-12A)。 该区域紧靠后顶叶皮层,并在整合物体的体感和视觉特性方面发挥作用。

生理学研究表明,S-II 在手部物体的触觉识别(立体视觉)、区分空间特征(例如形状和纹理)以及时间特性(例如振动频率)方面起着关键作用。 S-II 神经元的感受野比 S-I 大,覆盖手的整个表面,并且通常是双侧的,代表对侧和同侧手上对称的镜像位置。 如此大的感受野使我们能够感知一只手抓住的整个大物体的形状,使我们能够在工具接触手掌和不同手指时整合工具的整体轮廓。 双侧感受野使我们能够用两只手感知更大的物体,例如西瓜或篮球,在它们之间分担负荷。

S-II 神经元的大感受野也影响它们对运动和振动的生理反应。 S-II 神经元不将振动表示为与振荡频率相关的周期性尖峰序列,皮肤或 S-I 神经元的感觉纤维也是如此(图 19-9)。 相反,S-II 神经元抽象出振动刺激的时间或强度特性,以不同的平均速率针对不同的频率发射。 从时间编码神经元到速率编码神经元的类似频率依赖性转变是初级听觉皮层(第 28 章)中声音处理的基础,初级听觉皮层是与顶叶盖中的 S-II 皮层并列的大脑区域。

重要的是,S-II 神经元的放电率取决于主体的行为背景或动机状态。 在最近的优雅研究中,Ranulfo Romo 和他的同事比较了猴子在 S-I、S-II 和额叶不同区域对神经元振动刺激的反应,同时这些动物执行了两种选择的强制选择任务。 如果动物正确识别出两种振动刺激中哪一种频率更高,它们就会得到奖励。

S-I 中的神经元使用时间代码忠实地代表每个刺激的振动周期:它们与每个周期同相发射短暂的尖峰脉冲(图 19-9B)。 相反,S-II 神经元以非周期性尖峰序列响应第一个刺激,其中它们的平均放电率与振动频率直接或负相关(图 19-21A)。 他们对第二个刺激的反应更加抽象。 S-II 脉冲序列结合了两种刺激的频率(图 19-21B)。 换句话说,S-II 对振动的反应取决于刺激环境:相同的振动刺激可以引起不同的放电率,这取决于前一个刺激的频率是更高还是更低。

更有趣的是,Romo 的小组发现,S-II 中的神经元会将第一次刺激诱发的尖峰序列副本发送到前额叶和运动前皮层,以保存对该反应的记忆。 在第一次刺激结束后的延迟期间,这些额叶皮层区域的神经元会继续放电。 Romo 及其同事提出,当第二次刺激发生时,额叶中的这些区域将记忆信号发送回 S-II,从而改变 S-II 神经元对来自手的直接触觉信号的反应。 通过这种方式,先前刺激的感觉运动记忆会影响大脑中的感觉处理,从而使受试者能够对新到达的触觉刺激做出认知判断。

S-II 是通过岛叶皮层进入颞叶的通道。 内侧颞叶区域,尤其是海马体,对于外显记忆的存储至关重要(第 53 章)。 我们不会将进入神经系统的每一个触觉信息都存储在记忆中,而只会存储具有某些行为意义的信息。 鉴于选择性注意会修改 S-II 神经元的放电模式,S-II 可以决定是否记住特定的触觉信息。

\subsection{主动触摸参与后顶叶皮层的感觉运动回路}
Vernon Mountcastle、Juhani Hyvärinen 和其他人在 20 世纪 70 年代中期的研究表明,顶内沟周围的后顶叶皮层区域在运动的感觉引导中发挥重要作用,而不是在辨别触觉中发挥重要作用。 这些区域包括猴子的第 5 区和第 7 区以及人类的上顶叶小叶(Brodmann 区 5 和 7)和下顶叶皮层(第 39 和 40 区)。 这些和随后的研究表明,在伸手和抓握过程中,后顶叶皮层的神经活动与额叶皮层运动区和运动前区神经元的激活同时发生,并先于 S-I 的活动。 假设区域 5 和区域 7 参与手部动作的规划,因为后顶叶皮层接收会聚的中央和外围信号,使其能够在伸手和抓握行为期间将中央运动命令与体感反馈进行比较。 从 S-I 到后顶叶皮层的感觉反馈用于确认计划行动的目标,从而加强先前学习的技能或在发生错误时纠正这些计划。

预测手部动作的感官后果是主动触觉的重要组成部分。 例如,当我们看到一个物体并伸手去拿它时,我们会预测它应该有多重以及拿在手里应该有什么感觉; 我们使用这样的预测来启动抓取。 Daniel Wolpert 和 Randy Flanagan 提出,在主动触摸期间,运动系统控制传入大脑的体感信息流,以便受试者可以预测触觉信息何时应到达 S-I 并达到意识。 中枢和外周信号的汇聚允许神经元比较计划的和实际的运动。 从运动区到皮层体感区的必然放电可能在主动触觉中起关键作用。 它为后顶叶皮层神经元提供有关预期动作的信息,使它们能够学习新技能并顺利执行。


\section{大脑体感区的病变会产生特定的触觉缺陷}

S-I 皮层受损的患者难以对简单的触觉测试做出反应:触觉阈值、振动和关节位置感以及两点辨别力(图 19-22A)。 这些患者在更复杂的任务上也表现不佳,例如纹理辨别、立体视觉和视觉-触觉匹配测试。

手部触觉的丧失会产生明显的运动和感觉缺陷。 运动障碍不如感觉障碍明显,特别是在力和位置控制测试期间。 诸如接球或用指尖捏住小物体等探索性动作和技巧性任务也不正常。

手部感觉神经纤维的局部麻醉提供了一种直接的方式来理解触觉的感觉运动作用。 在正中和尺神经局部麻醉下,手部动作笨拙且不协调,抓握时发力异常缓慢。 随着触觉的丧失,人们完全依赖视觉来指导手。 失去触觉不会导致瘫痪或虚弱,因为许多熟练的动作都是可以预测的,必要时依靠感官反馈进行调整。 这些受试者的运动系统通过产生比必要更多的力来补偿触觉信息的缺失。

由于周围神经损伤或背柱损伤,长期、慢性的触觉功能丧失加剧了这些运动问题。 与某些疾病一样,传入神经阻滞会导致大脑传入连接发生重大变化。 在患有脱髓鞘疾病(例如多发性硬化症)的患者中,背柱中的有髓鞘传入纤维会退化。 在晚期梅毒中,背根神经节中的大直径神经元被破坏(tabes dorsalis)。 这些患者在触觉和本体感觉方面存在严重的慢性缺陷,但通常几乎没有温度知觉和伤害感受的丧失。 体感丧失伴随着运动缺陷:动作笨拙、协调性差和肌张力障碍。 类似的损伤发生在因中风或头部外伤或中央后回手术切除后 S-I 损伤的患者中。

后顶叶皮层病变的患者通常在简单的触觉测试中只有轻微的困难。 然而,他们在执行复杂的触觉识别任务时遇到很大困难,并且很少使用探索性和熟练的动作(图 19-22B)。 他们在与物体互动时表现出运动学缺陷,无法正确调整手的形状和方向以抓住它们,并且在伸手时手臂方向错误。 当将物体放在他们手中时,他们通常会用力过大,并且在被要求评估其大小和形状时无法正确引导手指。 这些缺陷在临床上被描述为“无用手”综合症(触觉失用症)。

由于疾病状态或创伤很少产生局限于一个局部脑区的损伤,因此对人类患者感觉缺陷的研究变得复杂。 出于这个原因,对动物实验控制病变的分析有助于理解在人类患者中观察到的感觉缺陷的病因。 例如,有楔形束损伤的猕猴在触觉辨别方面表现出慢性丧失,例如更高的触觉阈值、受损的振动觉和较差的两点辨别力。 在梳理、抓挠和操作物体期间,它们在控制精细手指运动方面也表现出重大缺陷。 通过抑制区域 2 的手部代表区域中的神经元,可以通过实验在猴子身上产生类似的熟练动作缺陷。

猴子皮层体感区的实验消融提供了有关这些区域功能的宝贵信息。 局限于 3b 区的小损伤会导致身体特定部位的触觉严重缺陷。 区域 1 中的病变会在评估物体纹理时产生缺陷,而区域 2 中的病变会改变区分物体大小和形状的能力。 当在幼年动物身上制造此类损伤时,对触觉功能的损害不太严重,这显然是因为在发育中的大脑 S-II 皮层可能会接管通常由 S-I 承担的功能。

去除猴子的 S-II 皮层会严重损害形状和质地的辨别力,并阻止动物学习新的触觉辨别力。 区域 2 或 5 的消融或抑制会导致粗糙度辨别缺陷,但被动触觉的其他变化很少。 然而,运动性能受损,因为这些动物误将手伸向物体,无法预先塑造手以熟练地抓住物体,并且由于缺乏触觉反馈而难以协调手指运动(图 19-23)。

在人类和猴子身上观察到的损伤之间的相似性是理解临床体感功能丧失的重要基础。 我们将在后面的章节中了解到,对猴子其他皮层区域的损伤研究也提供了对大脑高阶感觉和运动功能的深入了解。

\section{亮点}

1. 当我们用手探索一个物体时,大脑的很大一部分可能会被感官体验、它唤起的思想和情感以及对它的运动反应所吸引。 这些感觉来自参与前馈和反馈网络的多个皮层区域的平行动作。 

2. 第一次触摸时,外围感觉器官将物体分解成微小的部分,分布在大约 20,000 条感觉神经纤维的大量区域。 SA1 系统提供有关物体空间结构的高保真信息,这是形状和纹理感知的基础。 SA2 系统提供有关抓握和其他手部运动期间手部构造和姿势的信息。 RA1 系统传达有关手中物体运动的信息,使我们能够熟练地操纵它。 它们与 RA2 受体一起感知物体的振动,使我们能够将它们用作工具。 

3. 来自触觉感受器的信息通过脊髓的背柱纤维束、脑干和丘脑中的中继核以及皮质内通路的层级传递到意识中。 通过分析整个人群的活动模式,大脑构建了物体和手部动作的神经表征。 

4. 中枢通路的计算复杂且连续完成,从背柱核开始,经过丘脑和几个皮质阶段,终止于与记忆和感知相关的内侧颞叶皮质区域以及额叶运动区 调解自愿运动。 

5. 大脑对触觉的处理得益于每个中继所涉及的神经元的地形学、躯体组织。 一起受到刺激的相邻皮肤区域在中央继电器中在解剖学和功能上相互联系。 对触觉特别敏感的身体部位——手、脚和嘴——在大脑的大部分区域都有代表,反映了从这些区域传达的触觉信息的重要性。 

6. 中枢通路的另一个功能是将数千个神经元中对象属性的分解表示转换为少数神经元中复杂对象属性的综合表示。 代表相邻皮肤区域的神经元和皮质内抑制回路之间的会聚兴奋性连接使高阶皮质细胞能够整合物体的全局特征。 以这种方式,大脑的体感区域代表了特定类别对象的共有属性。 

7. 第三个功能是调节体感信息的传入流。 外围纤维传递的信息比任何时候都可以处理的多得多; 中枢神经通路通过选择信息传递给感知和记忆机制来进行补偿。 来自更高脑区的循环通路修改了触觉感受器提供的上行信息,从而使感觉信息流与以前的经验和任务目标相匹配。 

8. 最后,触摸系统提供了控制和引导运动所必需的信息。 顶叶和额叶皮层的感觉和运动区域之间的相互作用提供了一种神经机制,用于规划所需的动作、预测运动行为的感觉后果以及从重复经验中学习技能。

\section{选读}
\section{参考文献}

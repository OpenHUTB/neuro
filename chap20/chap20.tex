\chapter{痛觉}
% PDF所在目录: /data2/whd/win10/learn/neuro/neuro_神经科学原理_28_中枢神经系统的听觉处理.pdf

\section{有害损伤激活温度、机械和多模式伤害感受器}

\section{来自伤害感受器的信号被传送到脊髓背角的神经元}

\section{痛觉过敏既有外周起源也有中枢起源}

\section{四种主要的上行通路将伤害性信息从脊髓传递到大脑}

\section{几个丘脑核将伤害性信息传递给大脑皮层}

\section{疼痛的感知源于皮层机制并受其控制}
\subsection{前扣带回和岛叶皮层与疼痛感知有关}
\subsection{痛觉受伤害性和非伤害性传入纤维活动平衡的调节}
\subsection{大脑的电刺激产生镇痛}

\section{阿片肽有助于内源性疼痛控制}
\subsection{内源性阿片肽及其受体分布在疼痛调节系统中}
\subsection{吗啡通过激活阿片受体来控制疼痛}
\subsection{对阿片类药物的耐受和依赖是截然不同的现象}

\section{亮点}
\section{选读}
\section{参考文献}
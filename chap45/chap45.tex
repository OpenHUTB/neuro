\chapter{塑造神经系统}
% PDF所在目录: /data2/whd/win10/learn/neuro/neuro_神经科学原理_28_中枢神经系统的听觉处理.pdf

\section{神经管起源于外胚层}

\section{分泌信号促进神经细胞命运}
\subsection{神经板的发育是由来自组织者区域的信号诱导的}
\subsection{神经诱导是由肽生长因子及其抑制剂介导的}

\section{神经管的 Rostrocaudal 模式涉及信号梯度和二级组织中心}
\subsection{神经管在发育早期就区域化了}
\subsection{来自中胚层和内胚层的信号定义了神经板的 Rostrocaudal 模式}
\subsection{来自神经管内组织中心的信号会影响前脑、中脑和后脑}
\subsection{抑制性相互作用将后脑分成几个部分}

\section{神经管的背腹模式在不同的尾部水平涉及相似的机制}
\subsection{腹侧神经管由脊索和底板分泌的 Sonic Hedgehog 蛋白形成图案}
\subsection{背神经管由骨形态发生蛋白组成}
\subsection{背腹模式机制沿神经管的 Rostrocaudal 范围得到保护}

\section{局部信号决定神经元的功能子类}
\subsection{头尾位置是运动神经元亚型的主要决定因素}
\subsection{局部信号和转录回路进一步使运动神经元亚型多样化}

\section{发育中的前脑受内在和外在影响的影响}
\subsection{感应信号和转录因子梯度建立区域分化}
\subsection{传入输入也有助于区域化}

\section{要点}
\subsection{选读}
\subsection{参考文献}

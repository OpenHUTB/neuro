\chapter{脊髓中的感觉-运动整合}
% PDF所在目录: /data2/whd/win10/learn/neuro/neuro_神经科学原理_28_中枢神经系统的听觉处理.pdf

\section{脊髓中的反射通路产生肌肉收缩的协调模式}
\subsection{牵张反射可抵抗肌肉的拉长}

\section{脊髓中的神经元网络有助于反射反应的协调}
\subsection{牵张反射涉及单突触通路}
\subsection{伽马运动神经元调节肌肉纺锤体的敏感性}
\subsection{牵张反射还涉及多突触通路}
\subsection{高尔基肌腱器官向脊髓提供力敏感反馈}
\subsection{皮肤反射产生复杂的运动,起到保护和姿势功能的作用}
\subsection{中间神经元上感觉输入的收敛增加了反射对运动的贡献的灵活性}

\section{感觉反馈和下行运动命令在共同的脊髓神经元相互作用以产生自主运动}
\subsection{肌肉纺锤体感觉传入活动通过 Ia 单突触反射通路强化运动的中央指令}
\subsection{通过下行输入调节 Ia 抑制性中间神经元和 Renshaw 细胞协调关节肌肉活动}
\subsection{下行运动命令可能会促进或抑制反射通路中的传输}
\subsection{下行输入通过改变初级感觉纤维的突触效率来调节脊髓的感觉输入}

\section{随意运动的下行命令的一部分通过脊髓中间神经元传递}
\subsection{C3-C4 节段的本体脊髓神经元调节上肢运动的皮质脊髓命令的一部分}
\subsection{脊髓反射通路中的神经元在运动前被激活}

\section{本体感受反射在调节随意和自动运动中起着重要作用}

\section{脊髓反射通路经历长时变化}

\section{中枢神经系统的损伤会导致反射反应的特征性改变}
\subsection{脊髓下行通路中断经常导致痉挛}
\subsection{人类脊髓损伤导致一段脊髓休克期,随后出现反射亢进}


\section{要点}
\subsection{选读}
\subsection{参考文献}
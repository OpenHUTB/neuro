\chapter{前庭系统}

% 参考:https://www.dxy.cn/bbs/newweb/pc/post/40268362

\section{内耳的前庭迷路包含五个受体器官}
\subsection{毛细胞将加速刺激转化为受体电位}
\subsection{半规管感知头部旋转}
\subsection{耳石器官感知线性加速度}

\section{前庭中央核整合前庭、视觉、本体感受和运动信号}
\subsection{当头部移动时,前庭眼反射使眼睛稳定}
\subsection{前庭连合系统传递双边信息}
\subsection{联合半规管和耳石信号改善惯性感知并减少平移与倾斜的歧义}
\subsection{前庭信号是头部运动控制的重要组成部分}

\section{当头部移动时,前庭眼反射使眼睛稳定}
\subsection{旋转前庭眼反射补偿头部旋转}
\subsection{平移前庭眼反射补偿线性运动和头部倾斜}
\subsection{视动反应补充了前庭眼反射}
\subsection{小脑调节前庭眼反射}
\subsection{丘脑和皮质使用前庭信号进行空间记忆以及认知和感知功能}
\subsection{前庭信息存在于丘脑中}
\subsection{前庭信息广泛分布于大脑皮层}
\subsection{前庭信号对于空间定向和空间导航至关重要}


\section{临床综合征阐明正常的前庭功能}
\subsection{热量灌注作为前庭诊断工具}
\subsection{双侧前庭功能减退干扰正常视力}

\section{要点}
\subsection{选读}
\subsection{参考文献}


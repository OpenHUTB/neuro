\chapter{影响社会认知的障碍:自闭症谱系障碍}

\section{自闭症谱系障碍表型具有共同的行为特征}

\section{自闭症谱系障碍表型也有明显的认知异常}
\subsection{自闭症谱系障碍的社交沟通受损:心智失明假说}
\subsection{其他社会机制导致自闭症谱系障碍}
\subsection{自闭症患者缺乏行为灵活性}
\subsection{一些自闭症患者有特殊才能}

\section{遗传因素增加自闭症谱系障碍的风险}
\subsection{罕见的遗传综合症为自闭症谱系障碍的生物学提供了初步见解}
\subsection{脆性 X 综合征}
\subsection{权利综合症}
\subsection{威廉姆斯综合症}
\subsection{神经发育综合症提供对社会认知机制的洞察}

\section{自闭症谱系障碍常见形式的复杂遗传学正在得到阐明}

\section{遗传学和神经病理学正在阐明自闭症谱系障碍的神经机制}
\subsection{可以使用系统生物学方法解释遗传发现}
\subsection{自闭症谱系障碍基因已在多种模型系统中得到研究}
\subsection{尸检和脑组织研究提供了对自闭症谱系障碍病理学的洞察力}

\section{基础科学和转化科学的进展为阐明自闭症谱系障碍的病理生理学提供了途径}

\section{亮点}
\subsection{选读}
\subsection{参考文献}
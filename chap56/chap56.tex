\chapter{决策和意识}


\section{感知的鉴别需要一个决策规则}

\subsection{一个简单的决策规则是对证据表示阈值的应用}
\subsection{涉及深思熟虑的知觉决策模仿涉及认知能力的现实生活决策的各个方面}

\section{皮层感觉区域的神经元为决策提供嘈杂的证据样本}

\section{证据积累到阈值解释了速度与准确性的权衡}

\section{顶叶和前额叶联合皮层中的神经元代表一个决策变量}

\section{感知决策是从证据样本进行推理的模型}

\section{偏好决策使用了关于价值的证据}

\section{决策为理解思维过程、认知状态和意识状态提供了一个框架}

\section{意识可以通过决策的镜头来理解}


\section{亮点}
\subsection{选读}
\subsection{参考文献}
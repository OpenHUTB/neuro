\chapter{嗅觉和味觉:化学感觉}
% PDF所在目录: /data2/whd/win10/learn/neuro/neuro_神经科学原理_28_中枢神经系统的听觉处理.pdf

\section{一大群嗅觉受体启动嗅觉}
\subsection{哺乳动物共享一大类气味受体}
\subsection{不同的受体组合编码不同的气味}

\section{嗅觉信息沿着通往大脑的路径转化}
\subsection{气味由分散的神经元编码在鼻子中}
\subsection{嗅球中的感觉输入按受体类型排列}
\subsection{嗅球向嗅觉皮层传递信息}
\subsection{人类的嗅觉敏锐度各不相同}

\section{气味引发特征性先天行为}
\subsection{在两个嗅觉结构中检测到信息素}
\subsection{无脊椎动物嗅觉系统可用于研究气味编码和行为}
\subsection{嗅觉线索引起线虫的刻板行为和生理反应}
\subsection{嗅觉策略发展迅速}

\section{味觉系统控制味觉}
\subsection{味觉有五种反映基本饮食需求的亚模式}
\subsection{味蕾中发生促味剂检测}
\subsection{每种味觉形态都由不同的感觉受体和细胞检测}
\subsection{味觉信息从外围传递到味觉皮层}
\subsection{对味道的感知取决于味觉、嗅觉和体感输入}
\subsection{昆虫具有驱动先天行为的特定形态的味觉细胞}

\section{要点}
\subsection{选读}
\subsection{参考文献}
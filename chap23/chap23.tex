\chapter{中层视觉处理和视觉元素}
我们在第 21 章和第 22 章中已经看到,眼睛不仅仅是一个照相机,而是包含复杂的视网膜电路,该电路将视网膜图像分解为表示对比度和运动的信号。 这些数据通过视神经传送到初级视觉皮层,初级视觉皮层使用这些信息来分析物体的形状。 它首先识别对象的边界,由许多短线段表示,每个线段都有特定的方向。 皮层然后将此信息整合到特定对象的表示中,这一过程称为轮廓整合。

这两个步骤,方向的局部分析和轮廓整合,举例说明了视觉处理的两个不同阶段。 局部方向计算是低级视觉处理的一个例子,它与识别视野光结构的局部元素有关。 轮廓整合是中级视觉处理的一个例子,是生成统一视野表示的第一步。 在大脑皮层分析的最早阶段,这两个层次的处理是一起完成的。

一个视觉场景包含成千上万的线段和表面。 中级视觉处理涉及确定哪些边界和表面属于特定对象,哪些是背景的一部分(见图 21-4)。 它还涉及从表面反射的光的强度和波长区分表面的亮度和颜色。 反射光的物理特性既取决于照亮表面的光的强度和颜色平衡,也取决于该表面的颜色。 确定单个物体的实际表面颜色需要比较场景中多个表面反射的光的波长。

因此,中级视觉处理涉及将图像的局部元素组合成对物体和背景的统一感知。 虽然确定哪些元素属于一个单一对象是一个非常复杂的问题,具有天文数字的潜在解决方案,但大脑视觉回路中的每个中继都有内置逻辑,允许对元素之间可能的空间关系做出假设 . 在某些情况下,这些固有规则会导致视野中实际上不存在的轮廓和表面的错觉(图 23-1)。

视觉处理的三个特征有助于克服来自视网膜的信号的歧义。 首先,感知视觉特征的方式取决于它周围的一切。 例如,对点、线或表面的感知取决于该特征与场景中存在的其他事物之间的关系。 也就是说,视觉皮层中神经元的反应是依赖于上下文的:它取决于细胞感受野外的轮廓和表面的存在,以及细胞感受野内的属性。 其次,视觉皮层神经元的功能特性可以通过视觉体验或知觉学习来改变。 最后,皮层中的视觉处理受认知功能的影响,特别是注意力、期望和“感知任务”(积极参与视觉辨别或检测)。 这三个因素(表示场景的上下文或整套信号、皮质电路中依赖于经验的变化以及期望)之间的相互作用对于视觉系统对复杂场景的分析至关重要。

在本章中,我们将研究大脑对视觉场景中局部特征或视觉基元的分析如何与对更多全局特征的分析并行进行。 视觉基元包括对比度、线方向、亮度、颜色、运动和深度。 每种类型的视觉基元都受到中间级处理的综合作用。 具有特定方向的线被整合到物体轮廓中,局部对比度信息被整合到表面亮度和表面分割中,波长选择性被整合到颜色恒常性中,方向选择性被整合到物体运动中。

视觉基元的分析从视网膜开始,检测亮度和颜色,然后在初级视觉皮层继续分析方位、运动方向和立体深度。 与中级视觉处理相关的属性与从初级视觉皮层 (V1) 开始的视觉皮层中的视觉基元一起分析,它在轮廓整合和表面分割中发挥作用。 视觉皮层的其他区域专注于这项任务的不同方面:V2 分析与物体表面相关的属性,V4 整合关于颜色和物体形状的信息,而 V5——颞中区或 MT——整合空间中的运动信号(图 23– 2)。

\section{物体几何内部模型帮助大脑分析形状}

\section{深度感知有助于将物体与背景分离}

\section{局部运动线索定义目标轨迹和形状}

\section{上下文决定视觉刺激的感知}
\subsection{亮度和颜色感知取决于上下文}
\subsection{感受野属性取决于上下文}

\section{皮层连接、功能架构和感知密切相关}
\subsection{感知学习需要皮质连接的可塑性}
\subsection{视觉搜索依赖于视觉属性和形状的皮质表示}
\subsection{认知过程影响视觉感知}

\section{亮点}
\section{选读}
\section{参考文献}
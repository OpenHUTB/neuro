\chapter{感觉运动控制原理}
% PDF所在目录: /data2/whd/win10/learn/neuro/neuro_神经科学原理_28_中枢神经系统的听觉处理.pdf

\section{运动控制对神经系统提出挑战}

\section{可以自愿、有节奏或反射性地控制动作}

\section{运动命令通过感觉运动过程的层次结构产生}

\section{运动神经信号受前馈和反馈控制}
\subsection{快速运动需要前馈控制}
\subsection{反馈控制使用感官信号来纠正动作}
\subsection{身体当前状态的估计依赖于感觉和运动信号}
\subsection{预测可以补偿感觉运动延迟}
\subsection{感官处理可能因行动和感知而异}

\section{运动计划将任务转化为有目的的运动}
\subsection{许多动作都采用了刻板的模式}
\subsection{运动规划可以是降低成本的最佳选择}
\subsection{最佳反馈控制以任务相关的方式纠正错误}

\section{多个过程有助于运动学习}
\subsection{基于错误的学习涉及适应内部感觉运动模型}
\subsection{技能学习依赖于成功的多个过程}
\subsection{感觉运动表征约束学习}

\section{要点}
\subsection{选读}
\subsection{参考文献}


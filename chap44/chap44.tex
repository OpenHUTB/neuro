\chapter{睡眠和清醒}
% PDF所在目录: /data2/whd/win10/learn/neuro/neuro_神经科学原理_28_中枢神经系统的听觉处理.pdf

\section{睡眠包括交替的快速眼动睡眠和非快速眼动睡眠}

\section{上升的唤醒系统促进清醒}
\subsection{脑干和下丘脑的上行觉醒系统支配前脑}
\subsection{上行觉醒系统受损导致昏迷}
\subsection{由相互抑制的神经元组成的电路控制从清醒到睡眠和从非 REM 睡眠到 REM 睡眠的转变}

\section{睡眠受稳态和昼夜节律驱动器调节}
\subsection{睡眠的稳态压力取决于体液因素}
\subsection{昼夜节律由视交叉上核的生物钟控制}
\subsection{睡眠的昼夜节律取决于下丘脑中继}
\subsection{睡眠不足会损害认知和记忆}

\section{睡眠随年龄变化}
\subsection{睡眠回路的中断导致许多睡眠障碍}
\subsection{失眠可能是由唤醒系统抑制不完全引起的}
\subsection{睡眠呼吸暂停碎片睡眠和损害认知}
\subsection{发作性睡病是由食欲神经元的损失引起的}
\subsection{REM睡眠行为障碍是由REM睡眠麻痹回路故障引起的}
\subsection{不宁腿综合征和周期性肢体运动障碍扰乱睡眠}
\subsection{非 REM 异态睡眠包括梦游、梦话和夜惊}

\section{睡眠有很多功能}

\section{要点}
\subsection{选读}
\subsection{参考文献}

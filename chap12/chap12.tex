\chapter{直接门控传输:神经肌肉突触}

我们对控制大脑化学突触的原理的大部分理解是基于对骨骼肌细胞上运动神经元形成的突触的研究。 Bernard Katz 及其同事从 1950 年开始的 30 年间里程碑式的工作定义了突触传递的基本参数,并为突触功能的现代分子分析打开了大门。 因此,在我们研究中枢神经系统突触的复杂性之前,我们将研究更简单的神经肌肉突触中化学突触传递的基本特征。

早期的研究利用了不同物种的神经肌肉制剂提供的几个实验优势。 肌肉和附着的运动轴突很容易在体外解剖和维持数小时。 肌肉细胞足够大,可以被两个或多个尖端微电极穿透,从而能够精确分析突触电位和潜在的离子电流。 在大多数物种中,神经支配仅限于一个部位,即运动终板,而在成年动物中,该部位仅受一个运动轴突支配。 相比之下,中枢神经元接收许多分布在整个树突状乔木和体细胞中的会聚输入,因此单个输入的影响更难辨别。

最重要的是,在 20 世纪早期发现了介导神经和肌肉之间突触传递的化学递质乙酰胆碱 (ACh)。 我们现在知道神经肌肉突触的信号传递涉及一个相对简单的机制:从突触前神经释放的神经递质与突触后膜中的一种受体结合,即烟碱 ACh 受体。1 递质与受体的结合直接打开一个 离子通道; 受体和通道都是同一大分子的组成部分。 激活或抑制烟碱 ACh 受体的合成和天然药物已被证明不仅可用于分析肌肉中的 ACh 受体,而且可用于分析外周神经节和大脑中的胆碱能突触。 此外,此类配体可以是有用的治疗剂,包括治疗由 ACh 受体功能改变或基因突变引起的遗传性和获得性神经系统疾病。

\section{神经肌肉接头具有专门的突触前和突触后结构}


\subsection{膜通透性的局部变化导致突触后电位}

\subsection{神经递质乙酰胆碱以离散包的形式释放}


\section{单个乙酰胆碱受体通道传导全有或无电流}
\subsection{终板的离子通道可渗透钠离子和钾离子}
\subsection{四大因素决定终板电流}

\section{乙酰胆碱受体通道具有独特的特性,可将它们与产生肌肉动作电位的电压门控通道区分开来}
\subsection{递质结合在乙酰胆碱受体通道中产生一系列状态变化}
\subsection{分子和生物物理学研究揭示了乙酰胆碱受体的低分辨率结构}
\subsection{X 射线晶体研究揭示了乙酰胆碱受体通道的高分辨率结构}

\section{亮点}

\section{后记:端板电流可由等效电路计算}

\section{选读}

\section{参考文献}
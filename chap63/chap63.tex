\chapter{神经系统神经退行性疾病的遗传机制}

\section{亨廷顿病涉及纹状体的退化}

\section{脊髓延髓肌萎缩是由雄激素受体功能障碍引起的}

\section{遗传性脊髓小脑性共济失调症状相似,但病因不同}

\section{帕金森病是老年人常见的退行性疾病}

\section{普遍表达的基因受损后发生选择性神经元丢失}

\section{动物模型是研究神经退行性疾病的有效工具}
\subsection{小鼠模型重现神经退行性疾病的许多特征}
\subsection{无脊椎动物模型表现出进行性神经变性}

\section{神经退行性疾病的发病机制遵循多种途径}
\subsection{蛋白质错误折叠和降解导致帕金森病}
\subsection{蛋白质错误折叠触发基因表达的病理改变}
\subsection{线粒体功能障碍加剧神经退行性疾病}
\subsection{细胞凋亡和半胱天冬酶改变神经变性的严重程度}

\section{了解神经退行性疾病的分子动力学表明治疗干预的方法}

\section{亮点}
\subsection{选读}
\subsection{参考文献}